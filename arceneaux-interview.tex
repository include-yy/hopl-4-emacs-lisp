\documentclass[format=acmsmall, review]{acmart}
\settopmatter{printfolios=true,printccs=false,printacmref=false}

%% Metadata Information
\copyrightyear{2018}
%%\acmArticleSeq{9}
\acmNumber{HOPL} % CONF = POPL or ICFP or OOPSLA

%% Copyright
%%\setcopyright{acmcopyright}
%% \setcopyright{acmlicensed}
\setcopyright{rightsretained}
%%\setcopyright{usgov}
%%\setcopyright{usgovmixed}
%%\setcopyright{cagov}
%%\setcopyright{cagovmixed}
%% Paper history
\received{August 2019}

\newenvironment{question}{\begin{quote}\itshape}{\end{quote}}

\title{Interview: Joseph Arceneaux}

\begin{document}

\author{Michael Sperber}
\affiliation{%
  \institution{Active Group GmbH}
  \streetaddress{Hechinger Str.\ 12/1}
  \city{Tübingen}
  \country{Germany}
}
\email{sperber@deinprogramm.de}

\ccsdesc{Social and professional topics}
\ccsdesc{Professional topics}
\ccsdesc{History of computing}
\ccsdesc{History of programming languages}
\keywords{history of programming languages, Lisp, Emacs Lisp}

\maketitle

\noindent Joseph Arceneaux is an American software architect. In the early 1990,
Arcenaux was the maintainer of GNU Emacs, working first for the Free
Software Foundation, then for Lucid, Inc.  Lucid Inc.\ wanted to base
its Energize development environment for C++ on Emacs.  During
Arcenaux's tenture, The commercial interests of Lucid Inc.\ and those
of Emacs's original author, Richard Stallman, clashed, which finally
resulted in Lucid creating its own fork of Emacs, Lucid Emacs.

Arcenaux graciously agreed to be interviewed by Michael Sperber on
Mayx 30, 2019.  The purpose of the interview was to provide background
material for a paper on the evolution of Emacs Lisp for the
\textit{History of Programming Languages}  conference, specifically
what role Emacs Lisp played in the rift.
%
\begin{question}
Could you tell me how you got into software, tracing a line to how
you got involved with Emacs.
\end{question}
%
When I was 15 or 16, I was planning to become a veterinarian.  I
got out of high school early because I took a test, and then I took a
course in computer science and got really absorbed.  I went to what is
now the University of Louisiana, then the University of Southwest
Louisiana, and I had a Finnish professor who used Emacs and who
convinced me to use it.  By the time I got out of graduate school , I
was not just using Emacs but contributing code to the project, and I
became interested philosophically in the Free Software movement.

After I graduated I went to work doing research for the INRIA in
France.  Since I was a researcher, I had a fair amount of spare time
doing Emacs development, and I got to know Richard Stallman pretty
well.  I had him come over and give a couple of talks at INRIA.  This
was the first time Stallman rode a motorcycle when I took him to give
a talk.  He was scared to death.

We formed this professional
relationship, and I continued to contribute code to Emacs, and then my
research group started a startup company.  It was an X terminal on
steroids.  This was about the time X windows was really becoming
popular.  We had a our own hardware design, and our main windows guy
ported X windows to it.  I was the systems guy, so I wrote the BIOS,
and we adopted a new microkernel operating system, similar to Mach, called
Chorus. At some point, when our company grew, it became more and more
political, and at the same time, Stallman, whom we all called RMS,
because that was his login name, invited me to work for the Free
Software Foundation, and I thought this would be a great career move,
plus I was having a lot of fun writing code for Emacs. Emacs at that
point had been mostly been written by Stallman and contributors like
myself, with the main structure written by Stallman.  I found that the
structure of the code was such that it was mostly easy to understand
but especially easy to make extensions to.  Occasionally I would
submit something and RMS would say no, but most of time, he accepted
my stuff.  There were some clever things he did that required
pretty deep insight into how computers worked, but apart from two or
three of those examples, it was really easy to
write new code.

At the FSF, I spent a couple of years having a really good time at the
MIT AI lab.  RMS and I would have occasional arguments about how to do
something in Emacs.  And RMS is a really smart guy.  He was a
MacArthur genius award grantee.  Weird things would happen, like I'd
get this phone call, and it would be somebody from Norway inquiring
about chaos maths.  It turned out that RMS was this expert in chaos
maths.  So when we had arguments, he won a lot of arguments.  But I'd
say that only part of this was because he was a really smart guy.
Part of this was because he was very passionate.  I remember going
down the stairwell, with him behind me screaming, and then me
going back up chasing him up the stairwell with him covering his ears,
so he couldn't hear my argument.  So he's kind of an interesting guy.
I did win some arguments, though.

One of the big ones that came up was the window system.  So as I said
X windows was becoming really popular.  The X windows development
group was on the 2nd floor of our building.  It seemed natural to me
that we needed Emacs to become a full-blown X-windows-integrated
system.  RMS was adamantly opposed to this.  Not for a philosophical
reason, but he thought there were more important things to do.  A
version of Emacs that had been integrated with X10 worked pretty well,
but once the move to X11 was made, that didn't work anymore, and
nobody had bothered to port it (and, as I recall, the two systems were
rather incompatible).

There were some interesting technical questions, such as: At the time,
it wasn't computationally advisable to have characteristics associated
with every character in the buffer.  So a guy named Dan LaLiberte came
up with this system that he called "intervals".  Basically the idea
there was that you would have this shadow structure to the contents of
a buffer.  And it was essentially: From here to there the face that
we're using is this, and from there to over there is this other face.
A ``face'' was a structure that encompassed color, font, background, and
all
those characteristics.  So I took that stuff that he'd written and
prototyped, and I fleshed some stuff out and I added some other
features like caching, where we were in the buffer, etc.  I was doing
this in my
spare time while I was maintainer of Emacs, and integrating
contributions.  I started to push this as a major feature development
for Emacs.  RMS continued to be adamantly opposed to this.  So I
wasn't sure what to do, and I had a lot of discussions with the board
members of the FSF.  At some conference I met a French guy from Lucid,
Matthieu.  And partially because I spoke French, we got along.
Eventually Lucid offered me a contract to do this integration with X11
because they had a project to do an IDE for C and C++, and wanted to
use Emacs as the primary interface.  RMS and I had a big fight about
it because he didn't want me to do it.  My argument was "Hey, these
guys are going to basically pay my salary for a while, and I get this
done, and it won't cost the FSF anything."  And the board members
supported me on this, but RMS didn't want to do it.  So I went on to
do that, and RMS didn't talk to me for a year or so.  Also, I moved to
California after I got the contract at Lucid.

My contract with Lucid was in three parts.  The first one was the
most challenging, because it had to conform to certain specs,
performance-wise.  So if we had all of this visual annotation information,
and you did a search for some string, the requirement was that that
search should not be any slower than on any version of Emacs without
visual annotation.  So I passed that part.  I forget what the other
two parts were, but after the first phase of my contract, Lucid
changed the direction they wanted to go on, and tried to hire me as a
full-time employee.

\begin{question}
Do you remember why Lucid changed direction, or why their
requirements changed?
\end{question}
%
I don't remember exactly.  I feel that some of that was political.
But I don't remember the details.  I just remember that we had
arguments about which way to go.  By then I was back in contact with
RMS and I started checking in my code to the main repository of
Emacs.  And RMS was not always happy with this.  So there was a
conflict with this between Lucid and RMS about where we should go, and
I was trying to mediate that difference.

\begin{question}
But you were working mainly on a fork of Emacs at Lucid, or were
you always checking things into the Emacs mainline?
\end{question}
%
No, it was a fork.  And that was probably a big mistake.  If
git had been around back then, it wouldn't have been that big a deal.
The extent of changes was such that there would still have been
conflicts.  So we had a fork that was going to be version 19 of
Emacs.  Had I just continually incrementally checked in my changes, I
think that would have avoided a lot of political issues that came down
the road.

\begin{question}
I'm not sure that was an official position, but were you still been
maintainer of Emacs, or had you been and then went back to being that?
\end{question}
%
No. Eventually, the FSF hired a guy named Jim Blandy as the
official maintainer.  But again, this bifurcation became a big
political issue, and I was struggling to mediate between these two
somewhat contradictory sources of requests.  If I had just continued
in my role as maintainer and incrementally checked stuff into the
Emacs 19 fork, I think it would have avoided all of those issues.

\begin{question}
You said that for a long time RMS was opposed to the X11 windowing
stuff that ended up in Emacs 19.  But eventually Emacs 19 came out
with it.  So was that something that was forced onto the project or
did RMS come around and agree that it was a good idea?
\end{question}
%
As I recall, I think it was mostly that I convinced Jim Blandy that
this was really what we should do.  He and I worked on integrating all
of my changes into the Emacs 19 fork.  I think together we overcame
RMS's opposition.  This was 1992, I went to a conference to the Isle
of Jersey with RMS.  He basically had done an extensive review of my
code.  We talked about that, and when I saw that he had spent so much
time reviewing my code, I knew that he had been sold on the idea.

\begin{question}
So the various actors involved didn't just pull you in different
political directions but also technical directions.  But eventually
things came either to a head or an end, right?
\end{question}
%
Yes.  Lucid, when I declined their offer to hire me, essentially
took the position that they were going to create their own version of
Emacs.  They assigned a guy to be in charge of that.

\begin{question}
Probably Jamie Zawinski \ldots{}
\end{question}
%
I think so.  So he went off maintaining that, and they publicized
it, and it became known as Lucid Emacs, and later XEmacs.  And
meanwhile, there was some guy from Carnegie Mellon.  He was also
working on an X11 integration.  So for a while there were at least
three versions of Emacs that were integrated with X \ldots{}

\begin{question}
There was a windowed version of Emacs called Epoch at some point.
\end{question}
%
Yeah, that's it.

\begin{question}
After you declined Lucid's offer to hire you, did you continue
working for them on that contract basis, or was it done by then?
\end{question}
%
I continued working on that for a little while, but at some point,
especially after I went to that conference and talked to RMS, I went
back to mostly maintaining Emacs, and doing some of the other things.
I was also maintainer of GNU Indent.  My transition from Lucid was
gradual.  They didn't tell me at first that they were forking their
own version.  I was working out of my house, rather than the offices
of Lucid.

\begin{question}
I'm still trying to understand. I got that part where you worked on
the interval implementation, and how that performed to the speed that
Lucid needed.  Do you remember anything about the technical change of
direction that happened at Lucid?
\end{question}
%
As best as I can remember, it was about integrating with the
compiler to build a complete IDE.  So that had specific requirements
about how that should look like, how that should be.  As I recall, some of
those crossed a boundary between the vision I and RMS had for Emacs,
and the more commercial direction that Lucid wanted to go into.

\begin{question}
One thing that, looking back, surprised me was: You mentioned that
what Lucid was doing required quite significant performance.  They
wanted to stick annotations on every single letter of a C program or a
C++ program as you were editing it.
\end{question}
%
Not just C, but any text.

\begin{question}
Lucid ended up working on a development environment for C and C++,
and a compiler that would actually generate the data that would go
into these annotations.  What I'm getting at is: You got the
performance out of the redisplay and the buffer management.  I'm also
interested in Emacs Lisp, which was always a slow Lisp, it was never
very fast.  Was that never seen as a potential issue with Lucid that
Emacs Lisp would not be fast enough?
\end{question}
%
I remember doing an improvement to the string garbage collection.
But I would not say that the Lisp environment really entered the
picture due to performance because all of the annotation code was
done in C.  But over the years, there were several improvements.  RMS
and I came up with this algorithm for block-based storage management.
It basically replaced the Unix \texttt{sbrk()} system calls.  That greatly
improved the performance of Emacs in general.  So that was the C
world, hence I can't say what the impact was on Lisp, although the C
world was the base of the Lisp world, so I'm sure that had some
impact.

With respect to Emacs Lisp, I should add, one of the great pleasures
was that I got to work with some Lisp machines.  The Lisp machines
were really great because you could just do Meta-Dot (as in Emacs) and
you could see
the function definition of the thing that you'd just invoked, in the
very source code itself.

Richard Greenblatt, who invented the Lisp machine, had been absent from
the AI lab for years (see the book \textit{Hackers}, by Steven Levy for this story),
but---just after Steve Job's formation of NeXT, Inc---someone donated one
of their machines to us, and this attracted Greenblatt (aka 'rg') to come
visit us. We started hanging out, and (apart from performing some amazing
technical feats to help me out) he gave me much insight into how
Lisp came about.  Perhaps due to those conversations I changed my view on
the interaction between the C and Lisp world, and started doing more Lisp-y
things, like adding infinite minibuffer history, and some of that was all
in Lisp.

\begin{question}
  Excellent.  Thank you very much!
\end{question}
\end{document}