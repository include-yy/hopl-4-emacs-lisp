\documentclass[format=acmsmall, review=false, screen=true]{acmart}

\usepackage{booktabs} % For formal tables
%% \usepackage{alltt}
\usepackage[utf8]{inputenc}

%% Metadata Information
\copyrightyear{2018}
%%\acmArticleSeq{9}

%% Copyright
%%\setcopyright{acmcopyright}
%% \setcopyright{acmlicensed}
\setcopyright{rightsretained}
%%\setcopyright{usgov}
%%\setcopyright{usgovmixed}
%%\setcopyright{cagov}
%%\setcopyright{cagovmixed}

%% Paper history
\received{August 2018}

%% FIXME: While Richard prefers "Emacs Lisp" I find it unnecessarily verbose
%% and much prefer "Elisp" which is also very widely used.
\newcommand \Elisp {Emacs Lisp}
\newcommand \MAlign [1] {\begin{array}{@{}l@{}}#1\end{array}}

%% Document starts
\begin{document}
%% Title portion. Note the short title for running heads
\title{Evolution of Emacs Lisp}

\author{Stefan Monnier}
\affiliation{%
  \institution{Université de Montréal}
  \streetaddress{C.P.\ 6128, succ.\ centre-ville}
  \city{Montréal}
  \state{QC}
  \postcode{H3C 3J7}
  \country{Canada}}
\email{monnier@iro.umontreal.ca}
\author{Michael Sperber}
\affiliation{%
  \institution{Active Group GmbH}
  \streetaddress{Hechinger Str.\ 12/1}
  \city{Tübingen}
  \country{Germany}
}
\email{sperber@deinprogramm.de}


\input abstract

\ccsdesc{Social and professional topics}
\ccsdesc{Professional topics}
\ccsdesc{History of computing}
\ccsdesc{History of programming languages}

%%
%% End generated code
%%


\keywords{history of programming languages, Lisp, Emacs Lisp}


\maketitle

\tableofcontents

%% FIXME:
%%
%% - History is first and foremost about humans. The history in this article
%% is quite technical. I think it would be very interesting to know more
%% about the people behind ELisp and Emacs. Ex : Why did XEmacs fork?
%% Who decided in 199X how the language evolved? How does that compare to
%% the current sitution in 2018? How many programmers are involved? Does
%% functionnality XYZ required the joint efforts of 300 programmers or just
%% a summer intern at MIT? Lisp what shinny in 198X ... how is ELisp
%% perceived by its younger users today in 2018?
%%
%% - Most of the sections describe a new feature. Why not provide a before
%% and after short code snippet to illustrate why this features is important
%% and why should the reader know about it.
%%
%% - History is more than a compiled list of facts.  A good historian helps us
%% see the big picture, identify the main forces at play and also help us see
%% how things could have played out differently.  I think this article is
%% missing a narrative thread.  Why is feature X implemented in year 20XX.  Did
%% the previous features you talked made it feasable, did another
%% language featured this and people wanted to copy it, did the community
%% wanted to pioneer new technology, were the users complaining about speed?
%% Why was that feature implemented instead of the next one in the subsequent
%% years? Is there a common theme to features implemented in the 90s, 2000s,
%% 2010s?

\section{Introduction}

\Elisp{} is the extension language of the Emacs text editor.
In this sense, it is just a side-project of Emacs and might be overlooked as
a programming language.  But Emacs itself comes with more than a million
lines of \Elisp{} code, and yet more \Elisp{} is distributed separately from
Emacs in various Emacs Lisp Package Archives (ELPA).  If you additionally
consider that the majority of Emacs users have likely written a few lines of
\Elisp{} in their configuration file, it is arguably one of the most widely
used dialects of Lisp.

While in theory \Elisp{} exists independently from Emacs, its design,
implementation, and history are inextricably linked to that of Emacs.

%% FIXME: I feel like we should add something here, but not sure what.

\subsection{Emacs history}
\label{sec:emacs-history}

%% FIXME: What should we put here?

\subsection{Emacs early history}
\label{sec:emacs-early-history}

Emacs's original inception was as a set of macros for the TECO editor,
written by Richard Stallman with help from Guy Steele in
1976~\cite{Stallman2018-personal}.  This version of Emacs had no high-level
extension language.  Note that TECO and Emacs were not completely separate
entities in the sense that TECO's extension language was also Emacs's
extension language and that they evolved together, where TECO's extension
language was regularly extended to provide facilities then used by
Emacs~\cite{https://www.gnu.org/gnu/rms-lisp.html}.

This original Emacs grew popular in the lab and soon started to be
reimplemented in various other systems.  Several of those incarnations of
Emacs were written in Lisp, notably on the Lisp Machine (\emph{EINE} for
``EINE Is Not EMACS'') by Dan Weinreb and in
MacLisp~\cite{Moon1974,Pitman1983}, \emph{ZWEI} for ``ZWEI was EINE
initially'' by Dan Weinreb and Mike McMahon~\cite{Weinreb1979}), and by
Bernie Greenberg in 1978~\cite{Stallman2002}.  The ensuing possibilities
offered by Lisp for extending the editor were attractive to Richard
Stallman, who had worked on ZWEI~\cite{Stallman2018-personal} and decided to
write a (for him) second version of Emacs which would use Lisp as its
extension language like Greenberg's Multics Emacs, but would we written in
C so as to be more widely available.

\emph{Unix Emacs}, written in C by James Gosling in
1980/1981~\cite{Gosling1981}, preserved in the history books as
\emph{Gosling Emacs}, was the immediate predecessors of Emacs.
It featured an extension language called \emph{MLisp} or \emph{Mock Lisp},
which bears visual resemblance to \Elisp{}, but was a lesser member of the
Lisp family in the sense that it lacked \emph{cons} cells.

Richard Stallman started Emacs as we know it today in 1984 by replacing the
\emph{Mock Lisp} interpreter and data structures of \emph{Gosling Emacs}
with an \Elisp{} interpreter, adapting internal data structures to those of
this \Elisp{} interpreter.  When James Gosling later sold \emph{Gosling
  Emacs} to UniPress which in turn threatened Richard Stallman to stop
distributing his version of Emacs, forcing him to replace or rewrite the
rest of Emacs's code as well, such as the code responsible for the
redisplay, and to invent the GNU General Public License
\cite{http://www.free-soft.org/gpl_history/}, to try and ensure that users
of his code would never have to go through such an experience.

\subsubsection{Model of development}

Emacs is a Free Software project, developed by a loosely connected
%% FIXME: Not true at least for Gerd Möllman, maybe Joe?  Jim?
set of volunteers doing it mostly in their spare time, and several of them
do not consider themselves as computer professionals.  Development is not
really organized, in the sense that volunteers work on what they find
appealing rather than follow some agreed upon plan, and the overall
direction of evolution is in effect controlled by the head maintainer(s)
accepting or rejecting contributions, and via discussions in mailing-lists.

Richard Stallman was the original maintainer and even though several other
people have been official maintainers at various points in time, he always
stayed as a ``default maintainer'' when noone wanted to step up and has
always stayed very actively involved in Emacs's development.
%% %% FIXME: Do we really care about those details?
%% several hands:
%% %% FIXME: when?  I think it was around the time of collaboration with Lucid
%% Joseph Arceneaux was the first to take over maintenance,
%% %% FIXME: When?  Was it right after or Joe, or did Richard fill in-between?
%% then followed by Jim Blandy
%% %% FIXME Since when?  Did Richard fill in-between?
%% then followed by Gerd Möllman until the release of Emacs-22.1, after which
%% Richard came back as maintainer until 2008 when the maintenance was shared
%% between Chong Yidong and Stefan Monnier

\subsubsection{The great schism}

\subsection{The authors}

Neither of us was around when Emacs and \Elisp{} were designed, but we are
two researchers in programming languages with a keen interest and a lot of
experience in (X)Emacs and \Elisp{}, which made us well positioned to write
this article.  Stefan has been a core contributor to Emacs since
1999 and head maintainer from 2008 to 2015, while Mike has been a core
contributor to XEmacs since 199? and ??.


\subsection{Organization}

\Elisp{} has evolved in multiple strands and implementations over the
years, and thus its evolution did not happen along a single timeline.
Moreover, some aspects evolved over long periods of time.  To avoid
excessive interleaving of those aspects, we have organized the
top-level structure of the paper into chronological eras.  Within an
era, as a new topic is introduced, we usually follow that topic
chronologically to its conclusion, even if that means going beyond the
era where it started.

We trace the overall evolution of \Elisp{} in the software projects that
implemented it.  Its predecessors EINE (1976) and Multics Emacs (1978) were
themselves written in Lisp.  Unix Emacs (also known as Gosling Emacs), which
appeared in 1981, was written in C but included its own Lisp-like language.
We briefly describe those predecessors in Section~\ref{sec:prehistory}.

Emacs as we know it today itself started in 1984.
Section~\ref{sec:early-history} describes the driving motivation for
the early design and implementation of \Elisp.  The following Section
\ref{sec:base-language-design} traces the evolution of the base
language design, and Section \ref{sec:base-language-implementation}
its implementation.
Development continued at a high pace until about 1991.
Around that time, its development slowed down and
was overtaken by Lucid Emacs, later renamed XEmacs, whose development of
\Elisp{} we describe in Section~\ref{sec:xemacs}.  Eventually,
development of Emacs picked up again, and both co-evolved until about
2007.  We describe the relevant aspects of that evolution in
Section~\ref{sec:coevolution}.  After 2007, XEmacs development slowed
down, and we describe this post-XEmacs period in
Section~\ref{sec:post-xemacs}.

\Elisp{} was re-implemented in several other projects outside this
successions.  We briefly touch upon those projects in
Section~\ref{sec:alternative-implementations}.  Our conclusions are in
Section~\ref{sec:conclusion}.

\section{Prehistory}
\label{sec:prehistory}

Emacs's original inception was as a set of macros for the TECO editor,
written by Richard Stallman with help from Guy Steele in 1976~\cite{Stallman2018-personal}.  This
version of Emacs had no high-level extension language.  Arguably, the strongest
influence for \Elisp{} was MacLisp.  The design of the editing
primitives was influenced by Gosling Emacs's Mock Lisp.

\subsection{Gosling Emacs}

\emph{Unix Emacs}, written by James Gosling in
1980/1981~\cite{Gosling1981}, preserved in the history books
as \emph{Gosling Emacs}, was one of the immediate predecessors of
Emacs.  It featured an extension language called \emph{MLisp} or
\emph{Mock Lisp}, which bears visual resemblance to \Elisp{}.
MLisp featured function definitions via \texttt{defun}, as well as
many built-in functions (such as \texttt{eolp},
\texttt{forward-character}, \texttt{save-excursion}) whose names
survive in \Elisp{}.  Emacs contained some backwards-compatibility
support for MLisp until Emacs 21.2 in 2001.

MLisp was a quite limited language: It lacked cons cells and lists.
MLisp did have dynamic binding and local variables, but a peculiar
mechanism for passing arguments:  There were no named
parameters.  Instead, a program would invoke the \texttt{arg}
function: For example, \texttt{(arg 1)} would access the first
argument.  Moreover, argument expressions were essentially evaluated
in a call-by-name fashion by \texttt{arg}, and evaluation happened in
the dynamic environment of the callee.

The Mock Lisp primitives differed from ZWEI in that ZWEI primitives
would always take arguments to specify what text to operate on,
whereas in Mock Lisp primitives all operated at the current
%% FIXME: It's funny that Richard presents it this way, because the TECO
%% docs indicate that TECO already worked on the same principle, so
%% it could be argued that Gosmacs just followed the design of the original
%% Emacs.
point---the current ``cursor position.''  Stallman, who had worked on
ZWEI, found ZWEI's approach ``clumsy,'' and adopted Mock Lisp's basic
approach~\cite{Stallman2018-personal}.

%% FIXME Mike: Can we provide a code snippet doing the same thing but that
%% highlights the differences and similarity mentionned above?

\section{Early history}         % -1992?
\label{sec:early-history}

Following Greenberg's Multics Emacs, Richard Stallman decided to write a (for
him) second version of Emacs, which included \Elisp{} from the start.
As Greenberg's Emacs required a high-performance Lisp compiler to run
efficiently, Richard Stallman decided to reimplement the basis for the new
Emacs in C, with an integrated Lisp interpreter.  Buffer manipulation
and redisplay were written in efficient C, higher-level functionality
in Lisp.

The initial design of \Elisp{} was motivated by the pervasive
requirement of \emph{extensibility}; users ``must be able to
redefine each character''~\cite{Stallman1981}.  TECO and Gosling Emacs
featured small languages that were either too obscure or too weak to
support this vision.  Consequently, Richard Stallman took inspiration from
MacLisp, and \Elisp{} started as a real programming language with
%% FIXME: Add a summary or reference to that 10th rule?
powerful abstraction facilities, thus foregoing Greenspun's tenth
rule.

Moreover, Richard Stallman, made the design of \Elisp{}
embody and showcase the ideals of Free Software.  For example, not only
is it permitted to get and modify the source code, but every effort
is made to encourage the end-user to do so.  The resulting requirements had a profound
influence on the \Elisp{} language:
\begin{itemize}
\item The language should be accessible to a wide audience, so that as many
  people as possible can adapt Emacs to their own needs, without being
  dependent on the availability of someone with a technical expertise.
  This can be seen concretely in the inclusion in Emacs of the
  \emph{Introduction to Programming in Emacs Lisp}
  tutorial~\citep{ElispIntro} targeting users with no programming
  experience.  This has been a strong motivation to keep \Elisp{} on the
  minimalist side and to resist incorporation of many Common Lisp features.
\item It should be easy for the end-user to find the relevant code in order
  to modify Emacs's behavior.  This has driven the development of elements
  such as the \emph{docstrings} (see Section~\ref{sec:docstrings}) and more generally the self-documenting
  aspect of the language.  It also imposes constraints on the evolution of
  the language: the use of some facilities, such as \emph{advice}, is
  discouraged because it makes the code more opaque.
\item Emacs should be easily portable to as many platforms as possible.
  This largely explains why \Elisp{} is still using a fairly naive
  mark\&sweep garbage collector, and why its main execution engine is
  a simple byte-code interpreter.
\end{itemize}

\section{Base language design}
\label{sec:base-language-design}

The base language of \Elisp{} started out as a straightforward
variant of MacLisp~\cite{Moon1974,Pitman1983}.

Many basic special forms are identical: \texttt{defun},
\texttt{defvar}, \texttt{defmacro}, \texttt{let}, \texttt{let*},
\texttt{cond}, \texttt{if}, \texttt{function}, \texttt{catch}, \texttt{throw},
\texttt{unwind-protect}.  So are basic data structures: symbols,
\texttt{nil}, cons cells, arrays, and many familiar Lisp
functions.

The language used dynamic scoping where the latest binding of a variable is
simply stored in the \emph{value slot} of the corresponding symbol object,
resulting in simple and efficient variable lookups and let-bindings.
Symbols could be used to refer to variables.  In particular, the
\texttt{set} function, which performed assignment, accepted the
\emph{name} of a variable as its argument.  A symbol had an associated
\emph{property list}, essentially a dictionary mapping property
names (also symbols) to values.

Like MacLisp, \Elisp{} was (and still is) a
Lisp-2~\cite{SteeleGabriel1993}.  The namespaces for functions and
``ordinary values'' are separate, and to call a function bound to a
variable, a program must use \texttt{funcall}.  Also, symbols can be
used as function values.

Some specialized control structures were missing in the original
\Elisp{}, among them the \texttt{do} loop construct, and the
accompanying \texttt{return} and \texttt{go} forms for non-local
control transfer.  This was to reduce Emacs's memory footprint, in
order to make it work on Unix systems with one megabyte of memory~\cite{Stallman2018-personal}.

This section discusses some notable differences from and additions to
MacLisp.  Lambda expressions worked slightly differently, as discussed
in Section~\ref{sec:lambda}.  Section~\ref{sec:strings} discusses the
string representation (absent in MacLisp).
MacLisp-style backquote reader syntax was only added later to \Elisp{}---see Section~\ref{sec:backquote}.
A feature supporting easy extensibility was self-documentation
through \emph{docstrings}, described in
Section~\ref{sec:docstrings}.
Section~\ref{sec:interactive-functions} describes how Emacs ties
functions to interactive commands.
Section~\ref{sec:non-local-exits} describes \Elisp{}'s improvements
over MacLisp's error handling.
Section~\ref{sec:buffer-local-variables} describes buffer-local
variables, which tie the language to the editor.  One important
pragmatic aspect of the use of the language was extensibility of
existing code via hooks, discussed in Section~\ref{sec:hooks}.

\subsection{Lambda}
\label{sec:lambda}

Interestingly, \texttt{lambda} was not a keyword of the \Elisp{} language
(unlike in MacLisp) until around 1991 when it was added as a macro, early
during the development of Emacs-19.  In Emacs-18, anonymous functions were
written as quoted values of the form:
\begin{verbatim}
    '(lambda (..ARGS..) ..BODY..)
\end{verbatim}
While the \texttt{lambda} macro has made this quote unnecessary for almost
30 years now, many instances of this practice still occur in \Elisp{} code,
even though it prevents byte-compilation of the body.

Somewhat relatedly, only in 1993 did Lucid Emacs 19.8 import the
\verb|#'...| reader shorthand for \texttt{(function ...)} from MacLisp.  Emacs
followed suit the year after.

\subsection{Strings}
\label{sec:strings}

\Elisp{} included support for string objects from the beginning of course.
Originally, they were just byte arrays.

In 1992, during the early development of Emacs-19, this basic type was
extended by Joseph Arceneaux with support for \emph{text-properties}: each
char of a string can be annotated with a set of properties, which maps
property names to values, where property names can be any symbol.  This can
be used to carry information such as color and font to use when displaying
the various parts of the string.

XEmacs added a similar, though incompatible, feature to their strings at
around the same time, called \emph{extents}.  The incompatibility was due to
a fundamental disagreement about how those annotations should be propagated
when strings are manipulated: XEmacs's extents cover a contiguous span of
characters and they are objects in their own right with their own identity,
so they need to be duplicated when a substring is selected, with the new
extents not necessarily covering the same characters, and concatenation may
end up bringing two ``identical'' extents next to each other but cannot
merge them into one.  Richard Stallman felt like this was introducing
undesired complexities, and decided that Emacs's text properties should not
have identity and should apply to the characters of the string so there is
never a question of splitting or merging text properties.

With buffer-local variables~\ref{sec:buffer-local-variables}, this is one of
the very rare cases where the core \Elisp{} language has been extended with
a feature that specifically caters to the needs of the Emacs text editor, to
keep track of rendering information.  It is more generally useful, of
course, but turns strings into a much fancier datatype than in most
other languages.

Around 1994, support for arbitrary character sets was added to Emacs and
XEmacs, which required distinguishing between bytes and characters, and
hence changing its string objects.  Characters are represented with
a variable number of bytes, similar to \texttt{utf-8}, favoring a compact
representation at the cost of a slower random access.  In practice, random
access to strings is fairly rare, so it works rather well, but it does
impose a significant constraint: there is an indirection between the string
object and the bytes it holds because the \texttt{aset} primitive which
replaces one of the string's characters with another by side-effect will
sometimes change the string's length in bytes, which requires to relocate
the string's bytes elsewhere.

%% FIXME Mike: While reading the whole document, this part seemed rather
%% uninteresting.  I think we should either drop it, or spice it up
%% with information, stating at least in which context it was added.
In XEmacs, strings have a ``modified-tick'' that is bumped every time
a string is modified in-place.  This count can be retrieved with the
new function \texttt{string-modified-tick}.  The XEmacs codebase never
contained a user for this function, so the motivation for this is
unclear.  It is a straightforward parallel to the
\texttt{buffer-modified-tick} function that tracks modifications to
buffers, and which has many uses.

\subsection{Backquote}
\label{sec:backquote}

Curiously, \Elisp{} did not until 1994 include the reader syntax for
quasiquotation common to many other Lisps, including MacLisp~\cite{Bawden1999}.
Instead, the special forms that came with quasiquotation---backquote
(\verb|`|), unquote (\verb|,|) and unquote-splicing (\verb|,@|) were
the names of special forms.   Thus, what is usually written as:
%%
\begin{verbatim}
    `(a b ,c)
\end{verbatim}
%%
was written in early \Elisp{} as:
%%
\begin{verbatim}
    (` (a b (, c)))
\end{verbatim}
%%
The releases of Emacs-19.28 and XEmacs 19.12 in 1994 finally added the
proper reader support to generate the latter version from the first.
The reader had to rely on a heuristic for that, though, because \texttt{(`a)} is
valid in both syntaxes but means different things: It could either denote an
old-style backquote expression (denoting the backquoted symbol \texttt{a}) or
a single-element list containing the new-style backquoted symbol \texttt{a}.

%% Stefan: I can't see when/where it was actually declared obsolete,
%% but it clearly was made obsolete by the introduction of the new syntax.
So the old format, though obsolete, was still supported, and the
new format was only recognized in some cases; more specifically the new
unquote was only recognized within a new backquote, and the new
backquote was only recognized if it occurred within a new backquote or
if it did not immediately follow an open parenthesis.

Seeing how uses of old-style backquotes were not going away, in 2007
Emacs-22.2 introduced explicit tests and warnings to bring attention to uses
of old-style backquotes, while still keeping the actual behavior unchanged.

Then in 2012 with the release of Emacs-24.1, the behavior was changed so
that the old format is only recognized if it follows a parenthesis and is
followed by a space.  The main motivation for this change was the
introduction of the \texttt{pcase} macro where patterns can also use the
backquote syntax and are commonly placed right after a parenthesis so they
were otherwise mistaken for an old-style backquote syntax.

Finally in 2018 during the development of Emacs-27 the
old-style backquote syntax was removed.

\subsection{Docstrings}
\label{sec:docstrings}

An important feature of Emacs from the start was the idea of
\emph{self-documentation}~\cite{Stallman1981}, which originated with
the first, TECO-based Emacs~\cite{Stallman2018-personal}.  To that end, every
definition form in \Elisp{} can include documentation in the form of a
string after the signature:
%%
\begin{verbatim}
    (defun ignore (&rest _ignore)
      "Do nothing and return nil.
    This function accepts any number of arguments, but ignores them."
      nil)
\end{verbatim}
%%
This \emph{docstring} can be retrieved in various ways, in particular
through the user interface.  This idea was apparently first introduced in
the original Emacs implementation in TECO, then adapted to \Elisp{}.
Many languages have since adopted them, notably Common
Lisp~\cite{HyperSpec} and Clojure.

\subsection{Interactive functions}
\label{sec:interactive-functions}

The most direct method to make a function implemented in \Elisp{}
accessible to a user is to provide a keybinding for it.  This is not
realistic for all functions, however---there are too many of them.

A user can also invoke a function by typing \texttt{M-x}
\emph{function-name}.  This only makes sense for a certain subset of
all defined functions, namely \emph{interactive} functions, which
are specially marked with an \texttt{interactive} form at the
beginning of the body, like this:
%%
\begin{verbatim}
    (defun forward-symbol (arg)
      "Move point to the next position that is the end of a symbol.
    A symbol is any sequence of characters that are in either the
    word constituent or symbol constituent syntax class.
    With prefix argument ARG, do it ARG times if positive, or move
    backwards ARG times if negative."
      (interactive "^p")
      (if (natnump arg)
          (re-search-forward "\\(\\sw\\|\\s_\\)+" nil 'move arg)
        (while (< arg 0)
          (if (re-search-backward "\\(\\sw\\|\\s_\\)+" nil 'move)
              (skip-syntax-backward "w_"))
          (setq arg (1+ arg)))))
\end{verbatim}
The \texttt{interactive} form also has an optional string operand that
regulates how such a function receives its arguments.  (It can also be
an expression that is evaluated to produce a list of arguments.)  In the above
example, \verb|p| means that the function accepts a \emph{prefix
  argument}: The user can type \texttt{ESC} \emph{number} or
\texttt{C-u} (to specify powers of 4) prior to invoking the function,
and the number will be passed to the function as an argument, in this
case as \texttt{arg}.  (The \verb|^| (a more recent addition) has to do with
region selection with a pressed shift key.)

\subsection{Non-local exits}
\label{sec:non-local-exits}

Very early on, \Elisp{} featured a good set of primitives to handle
non-local exits.  Additionally to the \texttt{catch} and \texttt{throw}
primitives inherited from MacLisp, it came with \texttt{unwind-protect},
which was inherited from TECO, as well as an error handling system.

MacLisp's error-handling system was quite primitive, and had poor separation
of signalling and handling~\cite{Pitman2001}.  \Elisp{} instead featured
a \emph{condition system} inspired from the Lisp Machine although it did
not cover all the features of the Lisp Machine system, such as the ability
to recover from an error.

The \texttt{signal} function takes an \texttt{ERROR-SYMBOL} classifying the
exceptional situation and an additional \texttt{DATA} argument.  An error
symbol is a symbol with an \texttt{error-conditions} property that is a list
of condition names.  For example, the following invocation states that
a \texttt{pixmap} parameter is bound to an invalid argument, and
should have fulfilled the predicate \texttt{stipple-pixmap-p}:
%%
\begin{verbatim}
    (signal 'wrong-type-argument (list #'stipple-pixmap-p pixmap))
\end{verbatim}
%%
An invocation of \texttt{signal} escapes to the nearest
use of \texttt{condition-case} in the call stack, which dispatches on the
condition name.
These condition names can be used to dispatch on the condition with
the \texttt{condition-case} form which specifies a handler.
Here is an example:
%%
\begin{verbatim}
    (condition-case err
        (key-binding (this-command-keys))
      (wrong-type-argument
        <handle>))
\end{verbatim}
%%
This evaluates \texttt{(key-binding ...)}.  If a
\texttt{wrong-type-argument} is signalled during evaluation,
\texttt{<handle>} is evaluated, with \texttt{err} bound to a pair
consisting of the error symbol and the data argument passed to
\texttt{signal}.

Emacs comes with a set of standard errors, establishing a protocol
between signaling and handling code.

\subsection{Buffer-local variables}
\label{sec:buffer-local-variables}

Buffer-local variables are the prominent feature marking \Elisp{} as the
extension language of a text editor.  In Emacs, \emph{buffers} are objects
that store the contents of a file while it's being edited.  In addition to
the file's content, buffers store auxiliary information such
as the name of the associated file, the rules to use to highlight its
contents, the editing mode to use, the character encoding used etc.
Additionally, there is a global reference managed by Emacs called the
\emph{current buffer}, that determines the implicit target of
editing operations.

Making a variable \emph{buffer-local} associates the value of the variable
with the current buffer.  An assignment to a buffer-local variable
only affects dereferences with the same current buffer in effect.
Any variable can be made local to any buffer, so variables can take
different values in different buffers.  For example, the variable
\texttt{buffer-file-name} keeps the name of the file associated with the
corresponding buffer.  This variable typically has a different value in
every buffer.  In the absence of such a buffer-local assignment, a variable
is said to have its \emph{default} or \emph{global} value.

Variables can be both buffer-local and dynamically bound at the same time:
\begin{verbatim}
    (let ((buffer-file-name "/home/rms/.emacs"))
      (with-current-buffer "some-other-buffer"
        buffer-file-name))
\end{verbatim}
This example will not return \texttt{"/home/rms/.emacs"} but the
buffer-local value of \texttt{buffer-file-name} in the buffer
\texttt{"some-other-buffer"} instead because \texttt{with-current-buffer}
temporarily changes which buffer is current.

The presence of buffer-local values significantly complicates the
implementation of looking up, setting, and let-binding a variable, so the
code is optimized for the ``normal'' case of variables that have not been
made local to any buffer.

Over the years many bugs were fixed in the implementation of let-binding
buffer-local variables.  The most famous one was fixed in Emacs-21.1.
%% Stefan: When did it get fixed in XEmacs?
%% Mike: I don't see that it was ever "fixed".  What's the bug
%% anyway - any behavior I can imagine seems surprising to some.
%% Stefan: I explained the bug.
%% Mike: I can see that it was fixed in XEmacs, but can't see when
%% that was.
This bug affected code like
\begin{verbatim}
    (let ((buffer-file-name "/home/rms/.emacs"))
      ...
      (set-buffer other-buffer)
      ...)
\end{verbatim}
where the current buffer was different when the let-binding is entered from
when it is left\footnote{\texttt{set-buffer} is like
  \texttt{with-current-buffer} except that it is not scoped, so it takes
  effect until the next \texttt{set-buffer}.}; the bug was that this code
ended up ``restoring'' the value of \texttt{buffer-file-name} into the
wrong buffer when exiting the \texttt{let}.

\subsection{Hooks}
\label{sec:hooks}

One important aspect of the extensibility Richard Stallman originally
conceived for Emacs was the ability to make existing functions run
additional code without having to change them, so as to extend their
behavior.  Emacs supports this at well-defined points called
\emph{hooks}.

A hook is simply a variable bound to a list of functions.  The
\texttt{add-hook} function adds a function to such a list by
reassigning the variable.  (This makes use of the ability to refer to
a variable by its name symbol.)

Hooks are not a core language feature, but their use has been a
pervasive convention in Emacs from the start.  In particular, many
libraries run a hook when they are loaded to allow customization.
Also, modes generally run hooks to allow modifying their behavior.

The old TECO version of Emacs also allowed attaching hooks to variable
changes~\cite{Stallman1981}, but this feature was not provided in \Elisp{}
because Richard Stallman considered it a misfeature, which could make it
difficult to debug the code.  Yet this very feature was finally added to
\Elisp{} in 2018 in the form of \emph{variable watchers}, 
ironically meant to be used as debugging aides.

Of course, authors do not always have the foresight to place hooks where
users need them, so in 1992, the \texttt{advice.el} package was added to
Emacs-19, providing a \texttt{defadvice} macro duplicating a design
available in MacLisp and Lisp Machines, that allowed attaching code to
functions even if they do not run hooks.\footnote{``It is bad practice
  %% FIXME: I've often found people don't like footnotes very much.
  %% Should we try and eliminating them (by working them into the mainline text)
  %% or do we say that it's perfectly OK to use them?
  to make a Lisp program put advice on another Lisp
  program's function, because that is confusing.  When you see that the
  program calls \texttt{mumble}, it might take you hours before you think of
  checking whether it has advice.'' \cite{Stallman2018-personal}}

The way the \texttt{defadvice} macro gave access to function arguments did
not work with lexical scoping; furthermore time had shown that most of the
non-core features of \texttt{defadvice} were very seldom used, and even more
seldom used properly.  So in late 2012 a new package \texttt{nadvice.el} was
developed which provides the same core features but with a much simpler
design that tries to make better use of existing language features.  It was
released as part of Emacs-24.4.

\subsection{I/O}

One of the ways \Elisp{} distinguishes itself from most other programming
languages is in its treatment of input/output: rather than follow the usual
design based on some kind of file or stream object and primitives like
\texttt{open}/\texttt{read}/\texttt{write}/\texttt{close}, \Elisp{} only
offers coarser access to files via two primitive functions
\texttt{insert-file-contents} and \texttt{write-region} which transfer file
contents between a file and a buffer.  So all file manipulation takes place
by reading the file into a buffer, performing the desired manipulation in
this buffer, and then writing the result back into the file.

Since this approach does not extend naturally to interaction with external
processes or remote hosts, these are handled in a completely different way:
there are primitive functions that spawn a sub-process or open up
a connection to a remote host and return a so-called \emph{process} object.
These objects behave a bit like streams, with \texttt{process-send-string}
corresponding to the traditional \texttt{write}, but where the traditional
\texttt{read} is replaced by execution of a callback whenever data is
received from the sub-process or the remote host.

\section{Base language Implementation}
\label{sec:base-language-implementation}

At least initially, the \Elisp{} language was defined by its sole
implementation.  Various design aspects had impact on the users of the
language, and this section discusses some of them.
Section~\ref{sec:byte-code-interpreter} describes the byte-code
interpreter.
Section~\ref{sec:tco} discusses tail-call optimization.
Section~\ref{sec:bootstrap} touches on issues related to bootstrapping.
Section~\ref{sec:data-representation} describes the
essential data representation of the implementation.  The following
sections touch on other issues related to memory management:
Section~\ref{sec:stack-scanning} describes the evolution of scanning
GC roots. Section ~\ref{sec:heap-xemacs} describes changes in XEmacs
to simplify heap management.  Section~\ref{sec:tag-bits} describes
efforts in Emacs and XEmacs to reduce the tag bits.  Improvements to
the GC algorithm are described in Section~\ref{sec:gc-algorithms}.
Section~\ref{sec:image-dumping} describes the
image-dumping mechanism in Emacs and XEmacs and how it changed over time.
Sections~\ref{sec:debugger} and \ref{sec:profiler} document
debugging and profiling efforts, respectively.  Finally,
Section~\ref{sec:jit} describes recent efforts to add JIT compilation
to \Elisp{}.

\subsection{Byte-code interpreter}
\label{sec:byte-code-interpreter}

Emacs has two execution engines for \Elisp: The first is a very simple
interpreter written in C operating directly on the S-expression
representation of the code.
%% FIXME: This makes it sound like it was introduced in Emacs-16.56,
%% whereas I don't know when it was introduced: 16.56 is simply the first
%% revision I could get my hands on!
Some time before the release of Emacs-16.56 in July 1985,
a \texttt{byte-code} function was added, which interpreted its string
argument as a sequence of stack-based byte codes to execute, along with
a compiler, written in \Elisp{}, which translated \Elisp{} code to that
byte-code language.

While \Elisp{} is basically a ``run of the mill'' programming language, just
with some specific functions tailored to the particular use of a text
editor, this byte-code language is much less standard since it includes many
byte codes corresponding to \Elisp{} primitives such as
\texttt{forward-char}, \texttt{insert}, or \texttt{current-column}.

While a systematic study would likely reveal that the operations that should
deserve their own byte-code in modern \Elisp{} code are quite different, the
byte-code language of Emacs has changed very little over the years and is
still essentially the same as that of 1985.  The main changes were
those made to support lexical scoping; see Section~\ref{sec:lexical-scoping}.

\subsection{Tail-call optimization}
\label{sec:tco}

\Elisp{} does not optimize away tail calls.  With Scheme being familiar to
many \Elisp{} developers, this is a disappointment for many.
In 1991, Jamie Zawinski added an \texttt{unbind\_all} instruction to
the Lucid Emacs byte-code engine (which appears in both Emacs and
XEmacs to this day) that was intended to support tail-call optimization,
but never implemented the optimization itself.
%% Stefan: BTW, I remember seeing this unbind_all byte code and wanting to
%% implement the optimization, but I must admit that I could never figure out
%% how it's meant to be used.  I think Jamie added it without knowing either ;-)

There have been two patches developed independently and submitted to Emacs
maintainers in late 2012 (by Troels Nielsen first and Chris Gay two months
later) to optimize away tail calls in lexically-scoped byte-compiled code,
but so far none of them have made it into an official release.

The main reason for that is partly a chicken-and-eggs problem.  Tail-call
optimization (TCO) is largely incompatible with dynamic scoping, so it was
basically inapplicable until the introduction of lexical scoping in 2012;
furthermore, function calls are relatively expensive in the current
implementation of \Elisp{}.  Together, these two factors created a coding
style which favored the use of iteration over recursive definitions, which
in turn makes TCO rarely beneficial on existing code.

This said, there were other objections to those patches:
\begin{itemize}
\item They only affected byte-compiled code, and while it is expected that
  most code is byte-compiled before being executed, it's very common to run
  non-compiled \Elisp{} code, especially during development or debugging.
  So if \Elisp{} code started to rely on TCO, it would tend to cause problems
  when interpreted.  One alternative is to always byte-compile the code and
  get rid of the \Elisp{} interpreter, but that's a change that would have
  much further consequences.
  %% Note: TCO in general does not need to have this effect, notably
  %% MIT Scheme doesn't, so Mike softened the wording here.
  %% FIXME Mike: not sure I understand.  Do you mean to say that TCO does not
  %% necessarily reduce the stack usage?  That doesn't correspond to my
  %% understanding of TCO (unless you're thinking of stack in a literal
  %% way, where it may be unaffected for implementation that allocate
  %% their activation frames in the heap, but for those, I'd consider
  %% the linked list of activation frames to be the (heap-allocated) stack,
  %% and TCP still reduces that stack usage).
  %% FIXME Stefan: No, I just meant to say that it's not necessarily
  %% detrimental to debugging.  MIT Scheme uses you a ring buffer for
  %% the activation frames, giving *improved* debugging.  Maybe the
  %% two issues need to be detangled in the sentence.
\item The available implementations of TCO affect the behavior not only in
  terms of reducing the stack
  allocation and increasing performance, but it would also eliminate some
  activation frames from stack backtraces, which would be detrimental to
  debugging.  This is particularly true for tail-calls which are made to
  other functions, such as non-recursive tail-calls.
\end{itemize}

\subsubsection{Bootstrap}
\label{sec:bootstrap}

Since the \Elisp{} compiler is itself written in \Elisp{}, it requires
a form of bootstrap.  Until 2002 during the development of Emacs-21, Emacs's
development was done using the revision control system RCS~\cite{Tichy85}, and
changes were installed by logging remotely into an FSF-owned machine and
performing the commits there.  This machine also kept the (last) compiled
form of the \Elisp{} files and so naturally provided the needed compiled
files for bootstrap.  When development moved to CVS~\cite{Berliner90} to ease
collaboration between contributors, another solution was needed.

Of course, since Emacs also comes with a simple direct \Elisp{} interpreter,
bootstrapping is not very difficult, but several changes were needed
nevertheless because the previous use of pre-compiled files had hidden some
circular dependencies.  Of those, the only one that was not removed in
Emacs-21.1 is that some \Elisp{} code relies on the fact that some
functions are \emph{autoloaded}, and some of that code is used to build the
file that contains all those \texttt{autoload} declarations.  So instead,
the solution found was to keep a pre-built copy of that file in the
CVS repository.

\subsection{Data representation and memory management}
\label{sec:data-representation}

Emacs started with a data representation of its boxed data based on 32bit
words.  Those used a 7bit tag located in the most significant bits, 1 extra
``mark'' bit (see below), and 24bit of immediate data: either an integer or
a pointer.  Not all the possible 128 tags were actually used, but the 24
remaining bits where amply sufficient to represent the needed pointers and
integers for the typical available memory of the machines of that time.

The memory management used a simple mark\&sweep garbage collection
algorithm, and allocated objects by blocs of 4KB dedicated to a particular
kind of objects: one bloc each for cons cells, floats, symbols, markers, and
strings, all other objects were allocated directly with \texttt{malloc}.
To avoid fragmentation in the blocs of strings, those were compacted during
each GC.  The ``mark'' bit in each 32bit box was not used for the object
contained in the 32bit box.  Instead it was used so that cons cells could
occupy only 2 words: the extra bit needed to store the mark\&sweep's
\emph{markbit} of each cons cell was stored in the ``mark'' bit of the first
word (i.e. of the \texttt{car}).

Over time, this tagging scheme became problematic, since it limited to 16 MB
the size of the Lisp heap and to 8MB the size of files that could be edited.
The limit on file size is fundamentally linked to the maximum representable
integer since that is how buffer positions are represented.

So in 1995, with the release of Emacs-19.29,
the scheme was tweaked so that the tag was reduced to
3 bits, pushing the maximum file size to a more comfortable 128MB and the
maximum heap size to 256MB.  To reduce the tag to 3 bits, the less important
object types were placed into two groups: one group using the
\texttt{Lisp\_Misc} tag and another using the \texttt{Lisp\_Vector} tag.
The \texttt{Lisp\_Misc} tag was used for objects that could share the same
heap size as Lisp markers (6 words), and hence be allocated from the same
4KB bloc.  For some of those objects, it imposed a slight waste of space,
which was justified by the fact that objects using the \texttt{Lisp\_Vector}
tag had other extra costs: 2 words of header, plus the overhead of having
each object by allocated directly by \texttt{malloc}.

XEmacs implemented a similar change in 1993 with Lucid Emacs 19.8,
giving \Elisp{} 28-bit integers and a 28-bit address space, and
merged that data representation of Emacs 19.30 with the release of XEmacs 19.13, also in
1995.

\subsection{Scanning the stack}
\label{sec:stack-scanning}

Until Emacs-21, the mark phase of the GC was precise: the global roots were
explicitly registered, the GC knew all the types of Lisp objects and where
were the fields that could contain references, and the roots from the stack
were also explicitly registered into a singly linked list itself directly
allocated on the stack.

The cost of properly registering/unregistering stack references was
perceived to be high: it slowed down execution, both directly by adding
administrative code and indirectly by preventing some variables from being
kept in registers, and it was a source of bugs, especially since some code
tried to be clever and avoid registering local references under the
assumption that the GC could not be triggered at that particular point.
Similarly, the relocation of strings was a frequent source of hard-to-track
bugs because only the references known to the GC were properly updated, so
the programmer had to be careful not to keep unboxed or unregistered
references to a string at any point where a GC was possible.

To try and address those concerns, for Emacs-21.1, Gerd Möllmann changed the
string compaction code and implemented a conservative stack scan.
Strings are split into the string object itself, of fixed size and
non-relocatable, and the relocatable string data to which the code (almost)
never keeps a direct reference.  In order to find out if a given word found
on the stack might be a potentially valid reference to a Lisp object, it
keeps a memory map that records which regions of the memory contains which
kinds of Lisp objects.  This conservative stack scanning could be used
either in addition to the singly linked list of registered references, as
a kind of debugging aide, or replace it altogether.

Thanks to the interactive nature of Emacs and its opportunistic GC strategy
which ensures that the GC is often run when the stack is almost empty, the
slower conservative stack scanning and the potential false positives it
introduces have not been a problem.  The maintenance of the memory map,
implemented as a red-black tree, was hence the main cost of this new stack
scanning, which proved competitive with the previous scheme.  The previous
scheme was kept in use on some rare systems until Emacs-25.1, where all the
register/unregister code of stack references could finally be removed.

\subsection{Heap management in XEmacs}
\label{sec:heap-xemacs}

XEmacs has kept precise GC, and implemented a number of improvements
in its memory management.  In particular, Markus Kaltenbach and Marcus
Crestani implemented a scanning algorithm that used the memory-layout
descriptors that had been added to support the portable dumper.  (See
Section~\ref{sec:image-dumping}.)  Marcus Crestani also replaced the
allocator: The new, much simplified allocator eliminated the
distinction between objects that were allocated in blocks and those
allocated via \texttt{malloc}, and would make the decision
individually based on size.

In the process of these GC improvements, Crestani and Michael Sperber
also added \emph{ephemerons} to \Elisp{}~\cite{Hayes1997}.

\subsection{Squeezing the tag bits}
\label{sec:tag-bits}

Of course, a limit of 256MB was not actually comfortable.

XEmacs added a ``minimal-tagbits'' configure option with the release
of XEmacs 21.0 in 1998, yielding 31-bit integers and a 1GB maximum
file size.  The mark bits moved to the object headers, which were also
added to cons cells, making them 3-word objects.
The mark bit was removed from the boxes, and the number of
type tags was reduced to four: records (heap-allocated objects,
characters, even and odd fixnums.)  This became default with XEmacs
21.2 in 2002.

During the development of Emacs-22 in 2007, its tagging scheme was
similarly reworked: The mark bit was removed, and the tag bits were
moved to the least significant bits, allowing the Lisp heap to grow as
large as the full address space allowed.
Moreover, the \emph{markbit} of cons cells was moved to a separate bitmap
stored alongside each bloc of cons cells, which required allocating those
blocs on 4KB alignment boundaries.  This avoided the use of one extra word
per cons cell just to store the markbit, as was done previously for floats.
This same bitmap scheme
was of course also used for floats, thus reducing the typical heap size of
floats from 96 bits to 64 bits.  Reducing the heap size of floats to 64bits
and avoiding the use of 3 words for cons cells was not just motivated by
thriftiness but rather by the need to enforce that all objects be aligned on
a multiple of 8, so as to free the least significant 3 bits for use as
tag bits.

At that point, there was no illusion that a maximum file size of 256MB was
sufficient, but at the same time, the design of Emacs made it basically
impossible to view files larger than 4GB or edit files larger than 2GB (on
a 32bit system) anyway, no matter which tagging scheme we used, so there was
not a lot of room for improvement.

In Emacs-23.2 the tagging scheme was tweaked to use 2 tags for integers,
hence pushing the maximum file size to 512MB.

To cover the remaining space, a new compilation option
\texttt{--with-wide-int} was introduced in Emacs-24.1 to make the boxed data
use 64bit on 32bit systems.  This imposes a significant extra cost in terms
of space and time but makes it possible to edit files up to about 2GB.
When this compilation option is used, tag bits are placed in the most
significant bits again, so that the 32bit of pointers can be extracted at no
cost at all.

%% Mike: I find this of limited interest, as it doesn't really
%% impact the language.
%% Stefan: Hmm... OK, I changed it to talk a bit more about the
%% motivation for this change.  Do you still think we should remove
%% it?
%% Mike: I think it's fine now.

In Emacs-24.3, the allocation of objects using the \texttt{Lisp\_Vector} tag
(which is used for many more object types than just vectors) was modified:
instead of calling \texttt{malloc} for each such objects, they are now
allocated from ``vector blocs''.  The motivation was not that
\texttt{malloc} was too slow, but that the implementation of our
conservative stack scanning keeps track of every part of the Lisp heap
allocated with \texttt{malloc} in a balanced tree, so every such
\texttt{malloc} costs us an $O(N \textsf{log} N)$ operation plus a heap
allocation of an extra tree node, which was very costly for small objects
both in time and space.  The reason why it took until Emacs-24.3 to fix this
performance issue is that objects using the \texttt{Lisp\_Vector} tag were
historically not used in large numbers in early \Elisp{} code.  There were
two factors that changed this situation: first, over time the style of
\Elisp{} coding evolved and the use of \texttt{cl.el}'s \texttt{defstruct}
(which internally represents those objects as vectors) became much more
common, and second the closures used in the lexical-scoping feature of
Emacs-24.1 also use the \texttt{Lisp\_Vector} tag.

In Emacs-24.4, the representation of objects using the \texttt{Lisp\_Vector}
tag (which is used for many more object types than just vectors) was
improved so as to reduce their header from 2 words down to a single word.

In Emacs-27.1, the object representation was changed again: the distinction
between \texttt{Lisp\_Misc} and \texttt{Lisp\_Vector} was dropped by making
all objects use the \texttt{Lisp\_Vector} representation since it had been
improved sufficiently to be competitive with the special-cased
\texttt{Lisp\_Misc} representation.

%% FIXME Stefan: With the addition of bignums in the upcoming Emacs-27, there's
%% a chance that this compilation option won't be needed any more!

\subsection{New GC algorithms}
\label{sec:gc-algorithms}

During the early years of Emacs, the main complaints from users about the
simple mark\&sweep algorithm were the GC pauses.  These were solved very
simply in Emacs-19.31 by removing the messages that indicated when GC was in
progress.  Since then complaints about the performance of the GC have been
rare.  Most of them have to do with the amount of time wasted in the
GC during initialization phases, where a lot of data is allocated without
generating much garbage.

Over time, there have been several attempts to replace the GC.

%% FIXME: Gerd says: "It was just one of the things I have been working on, or
%% playing with, during the long time I had to wait to get the new redisplay
%% into Emacs, which was from 19.34 (?), when the new redisplay was more or
%% less ready (only variable-width fonts, AFAIR), to the end of the 20.x line"
Some time between 1996 and 1999, while waiting for Emacs-21 development to
start to be able to add his new redisplay engine, Gerd
Möllman worked on a generational incremental mostly-copying GC based on
a read barrier implemented using the operating system's VM primitives, but
it was never completed nor even actually integrated into Emacs's code.
Gerd gave up on this work when he realized that it was infringing on
a patent and hence wouldn't be distributable with Emacs anyway.
%% Its implementation did not get far enough to know its
%% performance characteristics and when Gerd stepped down as Emacs maintainer
%% in 2002 he stopped working on it and noone picked it up either.

During 2003, Dave Love worked on replacing Emacs's GC with Boehm's
conservative GC.  The effort went far enough to get a usable Emacs but it
was never completed, mostly because the early performance results were
disappointing.  It also showed that such a replacement is non-trivial.
The first issue is that various parts of Emacs's code assume that
a collection can only occur during \Elisp{} code execution and not during
heap allocation, and of course those assumptions are not explicitly recorded
in the code.  The second issue is that Emacs currently implements various
forms of ad-hoc weak references which need to be adapted to the more
%% FIXME: Is it worth going through the details of hash-tables's weakness
%% and markers's weakness (and their removal from undo lists)?
%% Mike: I think it might be, as it affects design.  I suggest leaving
%% it for the final version, however.
standard forms of weak references.

\label{sec:incremental-gc}
In XEmacs, GC pauses continued to be a perceived problem.  In 2005, Marcus
Crestani developed an incremental collector for XEmacs, again using
a VM-based write barrier.  It was released with XEmacs 21.5.21 in
2005~\cite{Crestani2005} and soon was turned on by default.  It
%% Damn!  Now I want even more that code in Emacs!  -- Stef
eliminates GC pauses from the user experience, and its asymptotic
performance is competitive with the old collector.

\subsection{Image dumping}
\label{sec:image-dumping}

An important feature of \Elisp{} has been the \texttt{dump-emacs}
function, which can be used to store a heap image of the running Emacs
into a file, which can later be restored.  This is crucial for Emacs's
usability, as it allows the editor to start up quickly, without
having to load and run all initialization code every time.

The implementation of \texttt{dump-emacs} started out as a set of highly
platform-specific C files that implemented a function called
\texttt{unexec}.  The \texttt{unexec} function turns the running process
back into an executable file.  However, the implementations of
\texttt{unexec} have been hard to write and required frequent maintenance.
For example, in 2016 the Glibc maintainers decided to make internal changes
to their \texttt{malloc} implementation which broke this functionality.

For XEmacs, in 1999, Olivier Galibert (based on initial work by Kyle
Jones) started writing a \emph{portable dumper} that would serialize
just the heap of the running XEmacs into a file, which could later be
\texttt{mmap}ed.  Galibert added explicit memory-layout descriptors
for all Lisp types to the system, which would later also benefit the
new incremental garbage collector.  (See
Section~\ref{sec:incremental-gc}.)  This required pervasive changes to
the C code, and so it took until 2001 for the first version of the
portable dumper to be released.

The Glibc announcement re-ignited interest in implementing a more portable
way to dump and restore Emacs's heap.  Some experiments were first made to dump
the heap in the form of a ``normal'' byte-compiled \Elisp{} file.
While this solution proved easy to implement, the time to load such a dumped
heap remained too high to be acceptable.  So the experiment only resulted in
the implementation of some improvements to the performance of the code to
read byte-compiled code.  In parallel, Daniel Colascione worked on
a different approach more like that of XEmacs's portable dumper.  The code
was basically finished, but Emacs maintainers so far have not accepted it
into the official release.  The main reason for that reluctance is that it
trades off the brittle platform-dependence of the old unexec code for more
code which partly duplicates some of the GC's code.

\subsection{Debugging}
\label{sec:debugger}

Debugging support was added very early to \Elisp: Emacs-16.56 already
included the \emph{backtrace debugger}, which suspends execution at the time
a \emph{condition} is signaled, showing the current stack backtrace and
letting you examine the state of the application and place breakpoints
before pursuing execution.

For most developers, this is still the main debugger for \Elisp{}.
Of course, it has seen various improvements over time, most of them
affecting only the user interface.  The main exceptions are:
in 1995 (for Emacs-19.31), it was refined so that the debugger is only
invoked for some conditions, making it possible for developers to keep this
debugger enabled all the time, without impeding normal use, and in 2012
(for Emacs-24.1) it was improved so as to be able to execute code in the
context of any activation frame, which was necessary to allow access to
lexically scoped variables.

In 1988, Daniel Laliberte developed another \Elisp{} debugger, called
Edebug.  It was included into Emacs a few years later during the early
development of Emacs-19.  This one works without any special support in the
interpreter; instead it instruments the \Elisp{} source code you want to
debug, such that running this code lets you step through this code and
displays the various values returned by the evaluation of each step.  We do
not know from where Daniel took this idea, but Edebug was probably the
inspiration for the Portable Scheme Debugger~\cite{Kellomaki93}, and the
same basic technique was used (and significantly improved) in
SML/NJ~\cite{Tolmach90}.

One interesting feature of Edebug is that in the presence of arbitrary
user-defined macros, it is generally impossible to correctly instrument
source code since Edebug cannot guess which arguments to a macro are normal
\Elisp{} expressions and which ones play a different role.  By default
Edebug works around this difficulty by leaving arguments to unknown macros
non-instrumented, which is safe but suboptimal.  To improve on this default
behavior, the macro author can annotate its macro with a \emph{debug
  specification} which describes the role of each argument using a kind of
grammar formalism, so Edebug can know which parts should be instrumented.

\subsection{Profiling}
\label{sec:profiler}

In 1992, during early development of Emacs-19, Boaz Ben-Zvi implemented the
\texttt{profile.el} package which implemented a fairly simple \Elisp{}
profiler all in \Elisp{}.  This implementation was based on instrumenting
a set of user-specified \Elisp{} functions by modifying their bodies in-place
to keep track of time spent in those functions.

In 1994, Barry A.~Warsaw implemented the \texttt{elp.el} package which took
a similar approach but without modifying functions's bodies, which was
brittle and inconvenient and only worked for functions defined in \Elisp.
Instead it replaced the instrumented functions with wrappers which counted
the number of calls, along with the execution time, and internally called
the original function's definition.  This package was included in
Emacs-19.29 and made \texttt{profile.el} obsolete.  Its implementation was
significantly reworked for Emacs-24.4 to make use of the new
\texttt{nadvice.el} package in order to add/remove instrumentation instead
of doing it in its own ad-hoc way.

Ben Wing added an \Elisp{} profiler to XEmacs 19.14 in
1996 by instrumenting entry points in the byte-code interpreter.
In profiling mode, the byte-code interpreter provides timing
information about function calls and allocation.

In early 2011, Tomohiro Matsuyama started implementing in Emacs's C code
a sampling-based profiler for \Elisp{}.  He finished the implementation as
part of Google's Summer of Code of 2012, and it was included in Emacs-24.3.
The main advantage of this profiler compared to \texttt{elp.el} is that it
does not require instrumentation, and it collects (partial) stack traces.
This means that not only the user does not need to know beforehand which
functions might be involved but it can show a an actual call tree.

\subsection{JIT compilation}
\label{sec:jit}
%% FIXME Mike: Why is this in this early section, even though it's a pretty
%% late effort that only affects Emacs, so it can be considered "post-XEmacs"?
%% Is that in preparation for the addition of some XEmacs-JIT efforts?
%% FIXME Stefan: No, I think the section is too lonely in the back,
%% which really is all about design now.  The supersection here is
%% about Base-language implementation.

The existing implementations of \Elisp{} are relatively inefficient, all
being based on fairly naive interpretation techniques.

\subsubsection{First attempt}
In 2004, Matthew Mundell developed the first JIT compiler for \Elisp.
This took byte-compiled \Elisp{} code and used the GNU Lightning library to
turn it into native machine code on the fly.
The speedup obtained reached a factor of about 2 in the best case, which
was rather disappointing, so it was not included into Emacs since the extra
maintenance burden was not considered justified.
An important reason for the disappointing performance is that it only
removed the immediate interpretation overhead, but did not affect function
calls nor was it able to remove redundant type checks.

\subsubsection{Second attempt}
Around 2012, Burton Samograd developed a second JIT compiler for \Elisp.
This took a similar path, but using GNU Libjit instead.  It was very
simplistic, turning each byte-code into a call to a C function.
The resulting performance was not more impressive, the author measuring
a 25\% speedup on a \texttt{raytracer.el} test.

\subsubsection{Third attempt}
In 2016, Nickolas Lloyd developed the third JIT compiler for \Elisp, again
based on GNU Libjit and based on a similar approach.  It improved on
Burton's implementation by open-coding most common byte-codes instead, which
avoided many C function calls, but it obtained comparable results most
likely because Libjit is not very good at optimizing its code and C function
calls aren't that costly.  but did get to the point of being stable enough
to be used globally.

\subsubsection{Fourth attempt}
In 2018, Tom Tromey took another stab at it, again using GNU Libjit.
Compared to Nickolas, it focuses exclusively on code using lexical binding,
which is likely to benefit more, and it implements additional optimizations
by getting rid at compile-time of all the manipulation of the Lisp stack.
In the best case it reaches a speed up factor of 3.5, which is not ideal,
but is a bit more respectable.  It can be used globally but is still
a very naive JIT compiler which requires further work to try and avoid
pathological behavior in some situations where the JIT compilation is
performed too eagerly, leading to significant slowdowns.

The discussion whether this last JIT compiler will be integrated as an
experimental feature into Emacs-27.1 is still on-going.

\section{XEmacs period}         % 1992-2007 ?
\label{sec:xemacs}
  
In 1991, Lucid Inc., a software development company based in Menlo
Park, Carlifornia, started a project called \emph{Energize}.
Energize was to be a C/C++ integrated development environment based on
Emacs~\cite{GabrielLetter}.  Lucid decided to use Emacs as the central
component of Energize.  At the time, the current version of Emacs was
18, which was still essentially a textual application.  The then
upcoming version of Emacs, Emacs 19, was to have a graphical user
interface and many other features that the developers at Lucid
considered essential for the development of Energize.  However, at the
time that Lucid needed Emacs 19, a release was not in
sight.\footnote{The first official release of Emacs 19, Emacs
  19.28, came out in on November 1, 1994.}

While Lucid at first tried to support and thus speed up the
development of Emacs 19, the required cooperation between Lucid and
the Free Software Foundation soon broke down.  As a result, Lucid
forked Emacs development, creating its own Emacs variant \emph{Lucid
Emacs}.\footnote{The first release of Lucid Emacs came out in April,
1992.}  Jamie Zawinski was the primary developer of Lucid Emacs.
In 1994, Lucid went bankrupt.  Sun subsequently wanted to ship
Lucid Emacs with their operating system, and ended up financing some
of the continued development of Lucid Emacs, and effected a name
change to the current \emph{XEmacs}.

The focus of Lucid Emacs was on providing a proper graphical user
interface.  As a result, most of the changes to \Elisp{} in Lucid
Emacs / XEmacs were to support the move from a TTY-based purely
textual model to a graphical model.

\subsection{Event and keymap representations}

One significant departure from Emacs in Lucid Emacs was the
representation of keymaps: Emacs, to this day, uses a transparent
S-expression representation for keymaps.
In early Emacs in a keymap was
simply a two-element list whose car is the symbol \texttt{keymap} and
whose second element is either a vector indexed by character code or an
%% This option was already present in Emacs-16.56!
\emph{association list}.
The current representation for keymaps in Emacs is richer but follows
largely the same design---here is an example~\cite{ELispManual2018}:
%%
\begin{verbatim}
    (keymap
     (3 keymap
        ;; C-c C-z
        (26 . run-lisp))
     (27 keymap
         ;; ‘C-M-x’, treated as ‘<ESC> C-x’
         (24 . lisp-send-defun))
     ;; This part is inherited from ‘lisp-mode-shared-map’.
     keymap
     ;; <DEL>
     (127 . backward-delete-char-untabify)
     (27 keymap
         ;; ‘C-M-q’, treated as ‘<ESC> C-q’
         (17 . indent-sexp)))
\end{verbatim}
%%
This creates problems with software evolution: While Emacs offered
constructors and mutators for keymaps, Emacs code could, in principle,
just use the tools for manipulating S-expressions for creating
them---\texttt{cons}, \texttt{rplaca}, \texttt{rplacd} etc.
Emacs-19 tried to keep such code working to some extent by merely
extending the previous list representation.
Eventually however such code would break in the face of representation
changes, but would not immediately trigger an error.  Furthermore, the
support for inheritance was fundamentally flawed until Emacs-24.1 where it
was extended to multiple-inheritance, but at the cost of a fairly significant
and delicate rewrite of the code.

The developers Lucid Emacs of Lucid Emacs foresaw these problems, and
instead made keymaps into an opaque datatype, forbidding manipulation via the
S-expression primitives and giving themselves a lot more implementation
freedom.  This allowed Lucid Emacs to evolve more rapidly the
representations of keymaps to cater to a richer set of input events
(including mouse events, for instance).

This step reflected a general difference in philosophy.  Lucid Emacs
also used opaque data types for case tables and input events, both of
which retain transparent representations in Emacs to this day.

\subsection{Character representation}
\label{sec:character-representation}

Another instance of a change in representation happened with the
release of XEmacs 20, the first release of XEmacs with support for
MULE (Multi-Lingual Emacs).

Previous versions of Emacs and XEmacs were inherently tied to an 8-bit
representation of characters.  Moreover, they had used strings not
only for representing text but also for representing key sequences.
In strings, the high bit represented ``meta,'' basically restricting
Emacs to ASCII.  Characters outside of strings could have more
modifiers in higher bits.

This situation was no longer tenable when multi-language support came
to XEmacs.  The work on MULE~\cite{Ohmaki2002} predates widespread
adoption of Unicode, and at the time XEmacs adopted MULE (around
1994), a number of other text encodings were still in use.  The MULE
character representation encoded a character as an integer that
represented two numbers, in the high and low bits respectively: One
represented the national character set, the other the associated
codepoint.

To enforce the separation between characters and their associated
encodings, XEmacs 20 made characters a separate data type.  XEmacs had
functions to convert between a character and its numerical
representation (\texttt{make-char} and \texttt{char-int}).  Generally,
\Elisp{} allows programs to mostly handle text as strings,
and avoid manipulating the numerical representation.  Making
characters an opaque type additionally discouraged the practice.

\subsection{C FFI}

As the \Elisp{} runtime was written in C, it was always possible to
add new \Elisp{} functions written in C to the system.  Those C
functions could also call \Elisp{} functions.

However, functions written in C originally lacked the dynamic nature
of \Elisp{}, as they had to be linked into the Emacs executable.
Starting in 1998, J.\ Kean Johnston added facilities to XEmacs
(released with version 21.2 in 1999), which allowed \Elisp{} code to
build and dynamically load shared libraries (called \textit{modules})
written in C into running editor and call the functions defined
therein.  Starting in 2002 with XEmacs 21.5.5, a number of such modules
were distributed with XEmacs, among them bindings for existing C
libraries such as Zlib, Ldap, PostgreSQL.

Even with modules in place, developers still had to create wrappers to
make existing C libraries accessible in \Elisp{}.  In 2005, Zajcev
Evgeny wrote an FFI for SXEmacs~\cite{SXEmacs}, a fork of XEmacs.
This FFI allows loading and calling existing C libraries directly,
without intervening wrappers, by declaring the type signatures of C
functions in \Elisp{}.  For example, the FFI allows using Curl like
this:
%%
\begin{verbatim}
    (ffi-load "libcurl.so")
    (setq curl:curl_escape
          (ffi-defun '(function c-string c-string int) "curl_escape"))
    (let* ((url "http://foo.org/please escape this<$!=3>")
           (str (ffi-create-fo 'c-string url))
           (len (ffi-create-fo 'int (length url)))
           (result (ffi-call-function curl:curl_escape str len))
      (ffi-get result))
\end{verbatim}

Richard Stallman refused to incorporate XEmacs's FFI into Emacs for fear
that it would open up a backdoor with which developers would be able to
legally circumvent the GNU General Public License (GPL) and thus link
Emacs's own code with code that does not abide by these licensing terms.
After many years of pressure on this issue (not just within the Emacs
project, since this affected several other GNU projects, most notably GCC),
a solution was agreed to, which was to implement an FFI that would only
accept to load libraries that came with a special symbol attesting that this
library is compatible with the GPL.  As a result, after a very long wait,
2016 finally saw the release of Emacs-25.1 with an FFI comparable in
functionality to that of XEmacs.  So far, we do not know of any publicly
available package which makes use of this new functionality, sadly.
But rumors indicate that it has been used in a few private projects, either
to link Emacs with another language or to extend Emacs with ad-hoc
functionality implemented in C for performance reasons.

\subsection{Aliases}

During the development of XEmacs 19.12, which was 1995, the first
official release of Emacs 19 appeared, Emacs-19.28.  Emacs had
implemented some XEmacs functionality, notably the support for
multiple open GUI windows.  XEmacs had called these windows
``screens,'' while Emacs called them ``frames.''  Compatibility with
Emacs was an important goal for the XEmacs developers at the time.
Consequently, they renamed the associated functionality.

To preserve compatibility for \Elisp{} code written for previous
versions of XEmacs, XEmacs introduced forms
\texttt{define-obsolete-functional-alias} and
\texttt{define-obsolete-variable-alias}.  The byte-code compiler would
emit warnings if these aliases were used, but still compile the code.

Emacs had long had a \texttt{defalias} form to declare function
aliases, on which the \texttt{define-\linebreak[0]obsolete-\linebreak[0]function-\linebreak[0]alias}
functionality could be based.\footnote{Curiously, \texttt{defalias}
  was elided from the Emacs code base in 1986 and reintroduced in
  1993.}  XEmacs 19.12 added a corresponding primitive form for variable
aliases, \texttt{defvaralias}, and functions \texttt{variable-alias}
and \texttt{indirect-variable} to examine the alias chains.

These additions were only merged into Emacs-22.1 in 2007.

\section{Emacs/XEmacs co-evolution}
\label{sec:coevolution}

Some aspects of \Elisp{} evolved in both Emacs and XEmacs, with both
versions borrowing design and code from the other.

\subsection{Performance improvements}

Both Emacs and XEmacs made various performance improvements, most of
which were merged back and forth between the two versions.

Jamie Zawinski and Hallvard Furuseth wrote a new optimizing
%% Stefan: we tend to use XEmacs a bit everywhere, but I wonder if we
%% should be careful to use "Lucid Emacs" when the timeframe corresponds
%% to before the name was changed to XEmacs (such as here).
%% Or maybe add a note in the intro about the fact that we often use XEmacs
%% to refer to what was then called Lucid Emacs.
%% WDYT?
%% Mike: You're right.
byte-compiler for Lucid Emacs, which, after some initial resistance, was
merged into the Emacs codebase by 1992.

Also around 1992, during the early development of Emacs-19, the
implementation of the byte-code interpreter was rewritten, and the
result ended up in both in Emacs and XEmacs.
As part of this rewrite, new object type for byte-compiled \Elisp{}
functions was introduced; before that, a byte-compiled function looked
like:
%%
\begin{verbatim}
    (lambda (..ARGS..) (byte-code "..." ...))
\end{verbatim}
%%
Then in Emacs-19.29 in an attempt to speed up loading of \Elisp{} packages
as well as reduce the memory use of Emacs processes, a mechanism was added
so that documentation strings as well as byte-code could be lazily fetched
from compiled \Elisp{} files.  This can introduce problems if the file is
modified while Emacs is running, so while this feature is always used
for documentation strings it's very rarely used for byte-code.

%% Stefan: any idea what kind of speedup this brought?
%% Mike: No, Ben Wing tended to submit massive patches where the details got
%% lost in the shuffle.

XEmacs added a just-in-time optimization pass to the byte code.  This
would perform some validity checks ahead (eliding them from the actual
execution), pre-compute stack use, make byte-code jumps relative
(saving a register), and optimize relative jumps with short offsets.
This effectively created an alternative byte-code dialect, which
XEmacs would convert back to the ``standard'' representation on
demand.
During the XEmacs 19 and much of the XEmacs 20 cycle, developers
avoided changing the byte-code format to make byte-code files
interchangeable between Emacs and XEmacs.  While no significant
changes were made to the byte-code format in XEmacs, the two
instruction sets eventually drifted and became incompatible.

In late 2009, the Emacs byte-code interpreter was modified by Tom Tromey to
implement token threading using GCC's \emph{computed goto} feature, when
available.  A patch for this feature had been submitted in May 2004 by
Jaeyoun Chung, but the speed improvement was not even measured at that time
so it had not raised much enthusiasm.  Tom's implementation was no better,
and the speed up was a meager 5\% but he pushed stronger for its inclusion,
which only happened with Emacs-24.3, in 2012.  The reason why the speed
improvement is a bit disappointing was not really investigated, but the
general consensus is that the byte-code interpreter is simply not very
optimized, so the relative cost of the \texttt{switch} is not as high as it
could (or, arguably, should) be.

\subsection{Unicode}

As Unicode~\cite{Unicode6} became universally adopted, Emacs and
XEmacs both supported the standard.  Emacs 21.1 supported the
\texttt{utf-8} coding-system, but it was not unified with other
charsets.  In 2001, Emacs 22.1 introduced a form of unification
between the Unicode charset and several other charsets.  XEmacs did
the same in 2001 with release 21.4.

As Unicode evolved a universal text representation and supplanted many
of the earlier encodings,, Emacs and XEmacs both started efforts to
replace the internal MULE representation by Unicode altogether.  This
appeared in Emacs 23 (2007) and XEmacs 21.5 (starting about 2010 in a
separate branch).  As a result, the integer representation of a
character in both Emacs and XEmacs is its Unicode scalar value.

\subsection{Bignums}

Somewhat surprisingly for Lisp, \Elisp{} had no support for
arbitrarily large integers (\emph{bignums}) for many years.
Integer range was restricted by word size on the underlying machine,
and representation changes over time have affected the exact range
available in \Elisp.
As a result, various functions dealing with numbers beyond the fixnum range
had to implement workarounds.  Notable are \texttt{file-attributes} (which
may use a pair of two fixnums for inode numbers, device numbers, user id,
and group id, and may use a float for the file size) and
\texttt{current-time}, which returns a list of numbers to encode the time.
Another place where the limited range of integers has caused friction has
been in the fact that it also limits the maximum size of file that can
be edited.

Moreover, \Elisp{} was used for more and more applications beyond
text editing, and also had to implement workarounds.  As a result,
Calc, an advanced calculator and computer algebra tool, which has
shipped with the Emacs distribution since 2001, had to implement
bignum arithmetic in Lisp.

Jerry James added bignums to XEmacs 21.5.18 in 2004, using the GMP
library~\cite{GMP}.  In Emacs, Gerd Möllman started work on adding support
for bignums via GMP around October 2001, but never finished it.  It's only
in August 2018 that Tom Tromey, with the help of Paul Eggert and several
other developers, finally added support for bignums to Emacs (again, using
GMP).

The support for bignums in XEmacs includes arbitrary-precision integers,
rationals, and floating point numbers and is optional at build time, so
while it is fairly complete, XEmacs's \Elisp{} programs still cannot rely on
bignum support.  Consequently, \texttt{file-attributes},
\texttt{current-time}, and Calc still do not take advantage of bignums.

In contrast, Emacs's bignum support is currently restricted to
arbitrary-precision integers but the feature is provided unconditionally by
bundling the \texttt{mini-gmp} library with Emacs for those systems where
GMP is not installed.  The lack of support for rationals and
arbitrary-precision floats is only a reflection of the lack of interest for
these features.  The support was made unconditional so that the code does
not need to keep alternate code paths for when bignums are not available.
As a result, \texttt{file-attributes} and Calc have been modified to use
native bignums (but not \texttt{current-time} which would need
arbitrary-precision rationals or floating points).

The introduction of bignums raises some design issues in \Elisp, as
previously integers were always unboxed.  This meant that the fast
\texttt{eq} only behaved differently from \texttt{eql} on floating point
numbers.  As a result, \Elisp{} could assume that, if two integers
represented the same number, \texttt{eq} would return true on them.
Bignums are heap-allocated, so the same is not necessarily true for two
bignums.  In XEmacs, \texttt{eq} can return \texttt{nil} in this case, and
this seems to have caused no serious problems.
%% FIXME Stefan: Update this when the discussion is over!
This issue is still being discussed with Emacs.


%% FIXME: \subsection{Customize}

%% ** Major Differences Between 19.11 and 19.12
%% ============================================


%% The new function `type-of' returns a symbol describing the type of a
%% Lisp object (`integer', `string', `symbol', etc.)

%% Symbols beginning with a colon (called "keywords") are treated
%% specially in that they are automatically made self-evaluating when
%% they are interned into `obarray'.  The new function `keywordp' returns
%% whether a symbol begins with a colon.

%% `get', `put', and `remprop' have been generalized to allow you to set
%% and retrieve properties on many different kinds of objects: symbols,
%% strings, faces, glyphs, and extents (for extents, however, this is not
%% yet implemented).  They are joined by a new function `object-plist'
%% that returns all of the properties that have been set on an object.

%% New functions `plists-eq' and `plists-equal' are provided for
%% comparing property lists (a property list is an alternating list
%% of keys and values).

%% The Common Lisp functions `caar', `cadr', `cdar', `cddr', `caaar', etc.
%% (up to four a's and/or d's), `first', `second', `third', etc. (up to
%% `tenth'), `last', `rest', and `endp' have been added, for more
%% convenient manipulation of lists.

%% New function `mapvector' maps over a sequence and returns a vector
%% of the results, analogous to `mapcar'.

%% New functions `rassoc', `remassoc', `remassq', `remrassoc', and
%% `remrassq' are provided for working with alists.

%% New functions `defvaralias', `variable-alias' and `indirect-variable'
%% are provided for creating variable aliases.

%% New macro `push' destructively adds an element to the beginning of a
%% list.  New macro `pop' destructively removes and returns the first
%% element of a list.

%% How did XEmacs bootstrap?
%% Strings with text-properties? No.

\subsection{Terminal-local and frame-local variables, specifiers}

In 1995, Emacs-19.29 added the ability to have \emph{frames} on
several different X11 servers at the same time.  XEmacs evolved
similarly.  This led to a requirement that certain aspects of display
should be local to a frame or an output device.  Emacs calls GUI
windows \emph{frames} and as with buffers, Emacs maintains a reference
to the \emph{current frame} which determines the implicit target of
GUI operations.  XEmacs also maintained \emph{devices} as part of
the display context, to distinguish, say, between different TTYs and
different X11 servers.

As buffer-local variables already allow settings that are sentitive to
context, Emacs furthered the analog by adding the notion of
\emph{terminal-local} variables, which are variables which take
different values depending on the X11 server (the \emph{terminal}) to
which the current frame belongs.  The set of terminal-local variables
is small and predefined in the C code; they are mostly used internally
to keep track of things like keyboard state; there is no way for
\Elisp{} programs to create others.

XEmacs chose a different route: Starting in 1995, Ben Wing (working
from a prototype by Chuck Thompson) implemented
\emph{specifiers}. which are objects that manage properties that
depend on a generalized notion of \emph{display
  context}~\cite{XEmacsLispRef1998}.  The first prototype
implementation was released with XEmacs 19.12.

A specifier's value (its \emph{instance}) depends on its
\emph{locale}, which can be buffer, a window, a frame, a device, or a MULE
character set, or certain properties of these.  For example,
\texttt{default-toolbar} is a specifier that could be created with:
%%
\begin{verbatim}
    (defvar default-toolbar (make-specifier 'toolbar))
\end{verbatim}
%%
(The \verb|'toolbar| is a type argument that the specifier system uses
for validation.)

\Elisp{} code could retrieve a specifier's value (its
\emph{instantiator}) like this:
%%
\begin{verbatim}
    (specifier-instance default-toolbar)
\end{verbatim}
%%
Code can modify a specifier with a value that is buffer-local as
follows:
%%
\begin{verbatim}
    (set-specifier default-toolbar <value> (current-buffer))
\end{verbatim}
%%
Emacs-20 in 1998 added the ability to set variable to a frame-local
value.  Contrary to terminal-local variables, any variable can be made
frame-local, and additionally, a variable can be both frame-local and
buffer-local at the same time.

In 2008, during the development of Emacs-23.1 several bugs were found and
fixed in corner case interactions between let-bindings and buffer-local and
frame-local variables (for example, when a variable is made buffer-local
between the moment a let-binding is entered and when it is left), and at
that occasion it was decided that variables should not be allowed to be both
buffer-local and frame-local.

The work on those bugs made it clear that the implementation of buffer-local
and frame-local bindings was too hard to follow, so in 2010 the
implementation was reworked to make the different possible states more
explicit in the code, and at the same time it was decided that frame-local
variables should be deprecated: while buffer-local variables are used
extensively in \Elisp{} and replacing them with explicit accesses to fields
or properties of buffer objects would make \Elisp{} code heavier,
frame-local variables were not in widespread use and could easily be
replaced by more traditional use of accessors to frame properties, making it
hard to justify the extra complexity in the implementation.  So in 2012 with
the release of Emacs-24.1, it became impossible to let-bind frame-local
variables any more, and in 2018 with the release of Emacs-26.1 frame-local
variables have been removed altogether.

\section{Post-XEmacs}           % 2007-now ?
\label{sec:post-xemacs}

Between 1991 and 2001, Emacs improved rather slowly compared to XEmacs.
But starting around 2001, Emacs's pace picked up again.  In 2008 Richard
Stallman stepped down (again) from the maintainership of Emacs, and the new
maintainers have proved more eager to make \Elisp{} evolve, whereas XEmacs
started to lose momentum starting about 2010.
%% Sadly, many improvements in XEmacs have never been
%% incorporated back into Emacs.

This section discusses some notable evolution of the design of
\Elisp{} during that time: lexical scoping
(Section~\ref{sec:lexical-scoping}), Common Lisp compatibility
(Section~\ref{sec:cl-clib}), generalized variables
(Section~\ref{sec:generalized-variables}), object-oriented programming
(Section~\ref{sec:oop}), native support for objects
(Section~\ref{sec:actual-objects}), generators
(Section~\ref{sec:generators}), concurrency
(Section~\ref{sec:concurrency}), inline functions
(Section~\ref{sec:inline-functions}) and various attempts at module
systems (Section~\ref{sec:module-system}).

%% FIXME: Section overview

\subsection{Lexical scoping}
\label{sec:lexical-scoping}

While Scheme was already about to get its second revision when Richard
Stallman started to work on GNU Emacs, and he obviously knew about
Scheme, being developed in a nearby office, \Elisp{} was mostly derived from
MacLisp and used exclusively dynamic scoping.  Dynamic scoping is used
extensively to add ``hidden'' configuration options to existing
functions without passing explicit arguments~\cite{Stallman1981}.

%% FIXME Stefan: I elided this because Stallman1981 lays out the
%% motivation quite clearly
%% FIXME Mike: Indeed, it explains well why dynamic scoping is provided,
%% but I think it's worth summarizing it here and it's worth adding
%% explanations for why it was the *only* binding provided (Stallman1981
%% only argues for the importance of having dynamic scoping).
%%
%% FIXME Stefan: Sure - but do we know that the text that was there
%% before is correct?  To me, Stallman1981 seems to say "we needed at
%% least dynamic scoping, and implementing lexical scoping in addition
%% would have been more work".  I may be wrong though, so feel free to
%% put the text back in.
%%
%% The motivation for this
%% decision is not completely clear but seems to include:
%% \begin{itemize}
%% \item Familiarity: Richard Stallman was more familiar with MacLisp than Scheme, and
%%   lexical scoping was new whereas dynamic scoping was the
%%   tried-and-tested alternative.
%% \item Hackability: Scheme's lexical scoping makes it impossible to access
%%   variables in other scopes, so it provides a strong form of abstraction,
%%   which is great in many cases, but can be annoying to the end user who just
%%   wants to do a quick and dirty hack.
%%   % FIXME: evidence?
%%   Similarly, Richard Stallman has often
%%   preferred to avoid the use of opaque datatypes, most famously for the
%%   representation of keymaps and characters.
%%   % FIXME: Needs to tie in with XEmacs above
%% \item Efficiency: It was perceived that closures would incur a cost that was
%%   not justified.  This was especially true in the context of the early Emacs
%%   which relied exclusively on interpretation to evaluate its \Elisp{} code.
%% \end{itemize}

Eventually, lexical scoping became the established standard in
the Lisp family in both Common Lisp and Scheme.
So of course, the question of adding lexical scoping to \Elisp{} has been
brought up many times.

The first implementation appeared quite early, in the form of the
\texttt{lexical-let} macro, which was part of the new \texttt{cl.el} by Dave
Gillespie \email{<daveg@synaptics.com>} introduced in 1993 and which
performed a local form of closure-conversion.
While this macro was used in many packages, it was never considered as
a good solution to the problem of providing lexical scoping.
The somewhat long name was likely a factor, but the reason was more probably
due to the fact that the code generated by the macro was less efficient than
equivalent dynamically-scoped code and was more difficult to debug because
the backtrace-based debugger showed you the gory details of the
macro-expansion rather than the corresponding source.  For these reasons,
\texttt{lexical-let} was only used in those particular cases where lexical
scoping was really beneficial.

Dynamic scoping had two main drawbacks in practice:
\begin{itemize}
\item The lack of closures.  Some packages circumvented the lack of closures
  by building lambda expressions on the fly with constructs like
  \texttt{`(lambda (x) (+ x ',y))}, which suffered from various problems
  such as the fact that macros within that closure were expanded late, and
  its code was not seen by the byte-compiler.  Emacs-23.1 introduced the
  curry operator \texttt{apply-partially} to cover similar use cases without
  those drawbacks.
\item The global visibility of variable names, requiring more care with the
  choice of local names.  The convention followed in Emacs to name all
  global variables with a package-specific prefix works well to avoid name
  conflicts, except in the presence of higher-order functions, like
  \texttt{reduce}, and it was also problematic in a few other cases such as
  in the byte-compiler: in order to emit warnings about the use of
  undeclared variables, the byte-compiler just tested whether that variable
  was already known to Emacs, which always returned true for those variables
  locally bound by one of the functions on the call stack, such as the
  functions in the byte-compiler itself.  So some code was made uglier with
  long local variable names in order not to interfere with other local
  bindings.  Worse: these ``solutions'' were never really complete.
\end{itemize}
%%
The only fully satisfactory solution to the desire for lexical scoping in
\Elisp{} was that it should be the scoping used by default by all binding
constructs, as is the case in Common Lisp.  But at the same time, there was
a non-negotiable need to preserve compatibility with existing \Elisp{} code,
although some limited breakage for rare situations could be tolerated.

The vast majority of existing \Elisp{} code was (and still is) agnostic to
the kind of scoping used in the sense that either dynamic or lexical scoping
of local variables gives the same result in almost all circumstances.  This was true of early
\Elisp{} code and has become even more true over time as the byte-compiler
started to warn about references to undeclared variables.  Warning about
unused variables would have probably pushed even more \Elisp{} code to be
agnostic.  But in any case, it seemed clear that despite the above, the
majority of Elisp packages relied somewhere on dynamic scoping.  So while
there was hope to be able to switch \Elisp{} to use lexical scoping, it was
not clear how to find the few places where dynamic scoping is needed so as
to avoid breaking too many existing packages.

In 2001, Matthias Neubauer implemented a code
analysis that, instead of trying to find the places where dynamic scoping is
needed, tries to find those bindings for which lexical scoping would not
change the resulting semantics~\cite{Neubauer01}.
This tool could have been used to
mechanically convert \Elisp{} packages to a lexically scoped version of
\Elisp{}, while preserving the semantics.  The plan with this approach
was to facilitate moving \Elisp{} code to Scheme eventually, but the
overall project was too large to ever be realized.

Around 2002, Miles Bader started working on a branch of Emacs with support
for lexical scoping.  His approach to the problem was to
have two languages: an \Elisp{} with dynamic scope and another with lexical
scope.  Each file was tagged to indicate which language was to be used, and
in turn, each function was tagged with which language it was using, so
functions using dynamic scope could seamlessly call functions with lexical
scope and vice versa.  This way, old code would keep working exactly as
before and any new code that wanted to benefit from lexical scoping would
simply have to add the corresponding ``\texttt{-*- lexical-binding:t -*-}''
at the beginning of the file.

The two languages were sufficiently similar that the new lexically scoped
variant only required minor changes to the existing interpreter.  But the
changes needed to support this new language in the byte-compiler were more
problematic, causing progress on this branch to be slow.
This branch was kept up-to-date with the main Emacs
development but the work was never completed.

It was only in 2010 that Stefan Monnier's student Igor Kuzmin
worked on a summer project in which he tried to solve the problem
differently: instead of directly adding support for lexical scoping and
closures to the single-pass byte-compiler code (which required significant
changes to the code), the idea was to implement a separate pass to perform
closure conversion as a pre-processing step.  This freed the closure
conversion from the constraints imposed by the design of the single-pass
byte-compiler, making it much easier to implement, and it also significantly
reduced the amount of changes needed in the byte-compiler, thus reducing the
risk of introducing regressions.

Two years later, Emacs-24.1 was released with support for lexical
scoping based on Miles Bader's \emph{lexbind} branch combined with Igor's
closure conversion.  The main focus at that point was to:
\begin{itemize}
\item minimize the changes to the existing code to limit incompatibility
  with existing \Elisp{} packages;
\item make sure performance of existing code was not affected by the
  new feature;
\item provide reliable support for the new lexical scoping mode, though not
  necessarily with the best performance.
\end{itemize}
%%
Changes to the byte code were introduced as part of the lexical scoping
feature that appeared in 2012 in Emacs-24.1, but were actually developed
much earlier, probably around 2003.  Until the introduction of
lexical-scoping, the stack-based byte-code only used its stack in the most
simple way, and did not include any stack operation beyond
\texttt{pop}/\texttt{dup}/\texttt{exch}, so to better support lexical
scoping where the lexical variables are stored on the stack, several
byte-codes were added to index directly into the stack, to modify stack
slots, and to discard several stack elements at once.

Performance of the new lexical scoping mode proved to be competitive with
the performance of the dynamic scoping mode except for its interaction with
the \texttt{catch}, \texttt{condition-case}, and \texttt{unwind-protect}
primitives whose underlying byte-codes were a poor fit, requiring run-time
construction of \Elisp{} code to propagate the lexical context into the body
of those constructs.
So in Emacs-24.4, new byte codes were introduced and
the byte-code compiler was modified to be able to make use of them.  Nowadays,
code compiled using lexical scoping is generally expected to be marginally
faster than if compiled with dynamic scoping.

\subsection{Eager macro-expansion} %Emacs-24.3?

The exact time at which a macro is expanded has never been clearly specified
in \Elisp{}.  Until Emacs-24, macro-expansion usually took place as late as
possible for interpreted code, whereas for byte-compiled code,
macro-expansion always took place during byte-compilation, with some notable
exceptions where the code was ``hidden'' from the byte-code compiler.  In the
byte-code compiler, the macro-expansion was also done ``lazily'' in that it was
done on the fly during the single pass of compilation.

In order to implement the separate closure conversion phase for Emacs-24,
this had to be changed so that the code is macro-expanded in a separate
phase before closure conversion and the actual byte-compilation, using a new
\texttt{macroexpand-all} function.
This caused some visible differences in corner cases where some macros ended
up expanded in code which earlier was eliminated by optimizations before
getting to the point of macro-expansion, but in practice this did not cause
any serious regression.

This use of the new \texttt{macroexpand-all} function was made yet a bit
more prevalent in Emacs-24.3 which applies it when loading
a non-compiled file, so that macro-expansion now happens ``eagerly'' when
loading a file rather than lazily when Emacs actually runs the code.  This eager
macro-expansion occasionally bumps into problematic dependencies (typically
in files which were never compiled), so it fails gracefully: if an
error is signaled during the macro-expansion that takes place while loading
a file, Emacs just aborts the macro-expansion and continues with the non-expanded
code as in the past, though not without duly notifying the user about
the problem.

Emacs-25.1 additionally fine-tuned these macro-expansion phases (both
while loading a file and while compiling them) according to the section
3.2.3.1 of the Common Lisp HyperSpec~\cite{HyperSpec}, so as to improve the
handling of macros that expand to both definitions and uses of
those definitions.

\subsection{Pcase}           %Released in Emacs-24.1

While working on the lexical-binding feature, Stefan Monnier grew
increasingly frustrated with the shape of the code used to traverse the
abstract syntax tree, littered with \texttt{car}, \texttt{cdr} carrying too
little information, compared to the kind of code he would write for that in
statically typed functional languages with algebraic datatypes.

So he started working on a pattern matching construct inspired by those
languages.  Before embarking on this project, he looked for existing
libraries providing this kind of functionality, finding many of them for
Common Lisp and Scheme, but none of them satisfying his expectations: either
the generated code was not considered efficient enough, or the code seemed
too difficult to port to \Elisp{}, or the set of accepted patterns was too
limited and not easily extensible.

So the \texttt{pcase.el} package was born,
being first released as part of Emacs-24.1, and used extensively in the part
of the byte-code compiler providing support for lexical binding.

Additionally to the \texttt{pcase} macro itself that provides a superset of
Common Lisp's \texttt{case} macro, this package also provides the
\texttt{pcase-let} macro, which uses the same machinery and supports the same
patterns in order to deconstruct objects, but where it is allowed to assume
that the pattern does match and hence can skip all the tests, leaving only the
operations that extract data.

After the release of Emacs-24.1, Stefan was made aware of Racket's
\texttt{match} construct~\cite{RacketReference2018}, which somehow eluded
his earlier search for
existing pattern matching macros and whose design makes it easy to define
new patterns.  The implementation of Racket's \texttt{match} could not be
easily reused in \Elisp{} because it relies too much on the compiler's
efficient handling of locally defined functions, but \texttt{pcase.el} was
improved to follow some of the design of Racket's \texttt{match}.
The new version appeared in Emacs-25.1 and the main resulting novelty was the
introduction of \texttt{pcase-defmacro} which can define new patterns
in a modular way, often using the new low-level pattern \texttt{app}.

\subsection{CL-lib}          %Released in Emacs-24.3
\label{sec:cl-clib}

While the core of \Elisp{} has evolved very slowly over the years, the
evolution of other Lisps (mostly Scheme and Common Lisp) has put pressure to
try and add various extensions to the language.  As it turns out, \Elisp{},
to a first approximation, can be seen as a subset of Common Lisp, so already
in 1986 Cesar Quiroz wrote a \texttt{cl.el} package which provided various
Common Lisp facilities implemented as macros.

Richard Stallman never wanted \Elisp{} to morph into Common Lisp, but he saw
the value of offering such facilities, so this \texttt{cl.el} package was
included fairly early on into Emacs, and has been one of the most popular
packages, used by a large proportion of \Elisp{} packages.  Yet, Richard did
not want to impose \texttt{cl.el} onto any Emacs user, so he imposed
a policy where the use of \texttt{cl.el} was restricted \emph{within} Emacs
itself.  More specifically, \Elisp{} packages bundled with Emacs were
restricted to limit their use of \texttt{cl.el} in such a way that
\texttt{cl.el} never needed to be loaded during a normal editing session.
Concretely, this meant that the only features of \texttt{cl.el} that could
be used were: macros and inlined functions.

The reasons why Richard did not want to use \texttt{cl.el} and turn \Elisp{}
into Common Lisp are not completely clear, but the following elements seem
to have been part of the motivation:
\begin{enumerate}
\item Common Lisp was considered a very large language back then, so in all
  likelihood it would have taken a significant effort to really make
  \Elisp{} into a reasonably complete implementation of Common Lisp.
\item Many aspects of Common Lisp design did not make consensus, as
  evidenced by the divide between Common Lisp and of Scheme.
  Richard disliked several aspects of Common Lisp's design, such as the use of
  keyword arguments, especially in low-level primitives like
  \texttt{mapcar}.
\item Some aspects of Common Lisp's design can incur an important efficiency
  cost, and Emacs already carried the stigma of \emph{eight megabytes and
    constantly swapping}, so there were good reasons to try and not make
  \Elisp{}'s efficiency any worse.
\item Keeping \Elisp{} small meant that users could participate in its
  development without having to learn all of Common Lisp.  When inclusion of
  Common Lisp features was discussed, Richard would often point out the cost
  in terms of the need for more, and more complex, documentation.
\item The implementation of \texttt{cl.el} was fairly invasive, redefining
  some core \Elisp{} functions.
\item Finally, turning \Elisp{} into Common Lisp would imply a loss of control,
  in that Emacs would be somewhat bound to Common Lisp's evolution and would
  have to follow the decisions of the designers of Common Lisp on most aspects.
\end{enumerate}
Over the years, the importance of the first two points has waned to some
extent.  Also the popularity of the \texttt{cl.el} package, as well as the
relentless pressure from Emacs contributors asking for more Common Lisp
features has also reduced the relevance of the third point.

XEmacs took the easy route on this and loaded \texttt{cl.el} into the
standard XEmacs image, starting with XEmacs 19.14 in 1996.  Emacs instead
took a longer road, where over the years, various macros and functions from
\texttt{cl.el} were found to be sufficiently popular to move them into
\Elisp{} proper:

\begin{itemize}
\item[1997] The release of Emacs-20.1 sees the move of the
  macros \texttt{when} and \texttt{unless} as well as the functions
  \texttt{caar}, \texttt{cadr}, \texttt{cdar}, and \texttt{cddr}.
\item[2001] Emacs-21.1 includes the hash-table functions, reimplemented in
  C, as well as the Common Lisp concept of \emph{keywords} (though only as
  objects, not as arguments).  Additionally the macros \texttt{dolist},
  \texttt{dotimes}, \texttt{push}, and \texttt{pop} are also added to
  \Elisp{}, which introduced some difficulties: in \texttt{cl.el} those
  macros included extra functionality which relied on parts of
  \texttt{cl.el} which we did not want to move to \Elisp{} proper,
  specifically \texttt{block}/\texttt{return} and generalized references.
  For that reason the macros added to \Elisp{} do not actually replace those
  of \texttt{cl.el}; instead when \texttt{cl.el} is loaded, it overrides the
  original macros with its own version.
\item[2007] Emacs-22.1 adds \texttt{delete-dups}, which provides a subset of
  \texttt{cl.el}'s \texttt{delete-duplicates}.
\item[2012] Emacs-24.1 adds \texttt{macroexpand-all} and lexical scoping,
  which obsoletes \texttt{cl.el}'s \texttt{lexical-let}.
\item[2013] Emacs-24.3 adds compiler macros, \texttt{setf} and
  generalized references.
\item[2018] To the \texttt{cXXr} functions incorporated in Emacs-20.1,
  Emacs-26.1 adds the remaining \texttt{cXXXr} functions.  The resistance
  against those was mostly one of style, since they tend to lead to poorly
  readable code.
\end{itemize}
%%
During the development of Emacs-24.3 the issue of better integration of the
\texttt{cl.el} package came up again.  The main point of pressure was the
desire to use \texttt{cl.el} \emph{functions} within packages bundled with
Emacs.  The main resistance from Richard Stallman was coming from the last point
above, but this time, a compromise was found: replace
the \texttt{cl.el} package with a new package (called \texttt{cl-lib.el})
which provides the same facilities but with names which all use the
\texttt{cl-} prefix.  This way, the \texttt{cl-lib.el} package doesn't turn
\Elisp{} into the Common Lisp language, but instead provides Common Lisp
facilities under its own namespace, leaving \Elisp{} free to evolve in its
own way.

On that occasion, some details of \texttt{cl.el}'s implementation were
reworked to be less invasive.  The main aspect was that \texttt{cl.el}
redefined Emacs's macro-expansion wholesale with its own implementation,
which incorporated support for \texttt{lexical-let}, \texttt{flet}, and
\texttt{symbol-macrolet}, so this was reworked in \texttt{cl-lib.el} to
provide those features while still using the standard macro-expansion code.

To encourage adoption of this new library, a forward compatibility version
of \texttt{cl-lib.el} for use on older Emacs and XEmacs versions was
released at the same time as Emacs-24.3.  Despite the annoyance of having to
use a \texttt{cl-} prefix, which caused some resistance to this new library,
the change has been surprisingly successful if we look at the proportion of
new packages which use \texttt{cl-lib.el} instead of \texttt{cl.el}.

\subsection{Generalized variables} %Released in Emacs-24.3
\label{sec:generalized-variables}

To facilitate the move to \texttt{cl-lib.el}, some frequently used
functionality from \texttt{cl.el} was moved directly to \Elisp{}.  The
most visible one is the support for \emph{generalized variables}, also
variously known as \emph{places}, \emph{generalized references}, or
\emph{lvalues}.  A generalized variable is a form that can be used as
an expression and an updateable reference.  The concept comes from
Common Lisp, and the Emacs implementation originally a part of
\texttt{cl.el}.  In both Common Lisp and Emacs, a number of special
forms take generalized variables as operands---in particular,
\texttt{setf}, which treats a generalized variable as a reference and
sets its value.

In Common Lisp, macros accepting a place can ask \texttt{get-setf-expansion}
to turn it into a list of five elements:
\begin{displaymath}
  (\textsl{VARS}~\textsl{VALS}~\textsl{STORE-VAR}~\textsl{STORE-FORM}
  ~\textsl{ACCESS-FORM})
\end{displaymath}
such that for example \texttt{(push \textsl{EXP} \textsl{PLACE})} turns into:
\begin{displaymath}
  \MAlign{
    \texttt{(let ((v \textsl{EXP}))} \\
    \;\;\;\MAlign{
      \texttt{(let* (\textsl{VARS} = \textsl{VALS})} \\
      \;\;\;\MAlign{
        \texttt{(let ((\textsl{STORE-VAR} (cons v \textsl{ACCESS-FORM})))} \\
        \;\;\;\texttt{\textsl{STORE-FORM})))}}}}
\end{displaymath}
This imposes a fairly rigid structure which, while general enough to adapt
to most needs, can be burdensome and leads to verbose code with a lot
of plumbing, both in the implementation of places and in the implementation
of macros which take places as arguments.

The original \texttt{cl.el} code followed this Common Lisp design.  But when
implementing the support for \texttt{setf} and friends in \Elisp{}, a fresh
new implementation of the concept was used.  The reasons for this new
implementation were:
\begin{itemize}
\item NIH syndrome: the existing code was hard to follow.
\item The previous code made use of internal helper functions from
  \texttt{cl.el} which we wanted to keep in \texttt{cl.el}, so some
  significant massaging was needed anyway.
\item the implementor of \emph{cl-lib} considered this part of Common Lisp's
  design ugly.
\end{itemize}
%% For example, if we want to define a place
%% of the form $\texttt{(if~\textsl{TEST}~\textsl{PLACE1}~\textsl{PLACE2})}$
%% the above \texttt{push} will inevitably end up with one of two
%% possibilities:
%% \begin{itemize}
%% \item Check twice whether \textsl{TEST} was nil or not: once within
%%   \textsl{ACCESS-FORM} and once within \textsl{STORE-FORM}.
%% \item Do the check once within \textsl{VALS} to return a pair of an ``access
%%   function'' and a ``store function'' which are then called via
%%   \texttt{funcall} within \textsl{ACCESS-FORM} and \textsl{STORE-FORM}.
%% \end{itemize}
%%
So the reimplementation uses a different design: instead of a five element
list, a \emph{place} maps to a single higher-order function.
This higher-order function takes as its sole argument a \textsl{DO} function
of two arguments, the \textsl{ACCESS-FORM} and the \textsl{STORE-FUNCTION}.
For example, the \texttt{push} macro could be naively implemented as:
\begin{verbatim}
    (defmacro push (EXP PLACE)
      `(let ((x ,EXP))
         ,(funcall (gv-get-place-function PLACE)
                   (lambda (ACCESS-FORM STORE-FUNCTION)
                     (funcall STORE-FUNCTION `(cons x ,ACCESS-FORM))))))
\end{verbatim}
instead, \Elisp{} provides a macro \texttt{gv-letplace} which lets us
write the above as:
\begin{verbatim}
    (defmacro push (EXP PLACE)
      `(let ((x ,EXP))
         ,(gv-letplace (ACCESS-FORM STORE-FUNCTION) PLACE
            (funcall STORE-FUNCTION `(cons x ,ACCESS-FORM)))))
\end{verbatim}
This design generally leads to cleaner and simpler code, and we can easily
provide backward compatibility wrappers for most of Common Lisp's
primitives.

Any 5-tuple representation of a place can easily be turned into
a corresponding higher-order function.  The reverse is not true, however, so
this design precludes compatibility with Common Lisp and
\texttt{cl.el}'s \texttt{get-setf-expansion}, which must produce the
five values described above.
Breaking compatibility with \texttt{get-setf-expansion}
was of course
a downside, but in practice this function was almost never used outside of
\texttt{cl.el} itself so very few packages were impacted by
this incompatibility.

%% %% FIXME: The above is already borderline for a *history* of Elisp, but
%% %% I think the examples below are past the line.
%% For reference here is the definition of the \texttt{nthcdr} place, in Emacs-23:
%% \begin{verbatim}
%% (define-setf-method nthcdr (n place)
%%   (let ((method (get-setf-method place cl-macro-environment))
%%         (n-temp (make-symbol "--cl-nthcdr-n--"))
%%         (store-temp (make-symbol "--cl-nthcdr-store--")))
%%     (list (cons n-temp (car method))
%%           (cons n (nth 1 method))
%%           (list store-temp)
%%           (list 'let (list (list (car (nth 2 method))
%%                                  (list 'cl-set-nthcdr n-temp (nth 4 method)
%%                                        store-temp)))
%%                 (nth 3 method) store-temp)
%%           (list 'nthcdr n-temp (nth 4 method)))))
%% \end{verbatim}
%% And here is the corresponding definition in Emacs-24:
%% \begin{verbatim}
%% (gv-define-expander nthcdr
%%   (lambda (do n place)
%%     (macroexp-let2 nil idx n
%%       (gv-letplace (getter setter) place
%%         (funcall do `(nthcdr ,idx ,getter)
%%                  (lambda (v) `(if (<= ,idx 0) ,(funcall setter v)
%%                            (setcdr (nthcdr (1- ,idx) ,getter) ,v))))))))
%% \end{verbatim}
%% While the intensive use of higher-order functions may be a bit of an
%% obstacle for some programmers, this code has much less plumbing, and it is
%% much easier to see that \texttt{n} and \texttt{place} are evaluated in the
%% proper order.

\subsection{Object-oriented programming} %Emacs-25.1
\label{sec:oop}

While \texttt{cl.el} early on (originally developed in 1986 by Cesar Quiroz
\email{<quiroz@cs.rochester.edu>}, and included in Emacs-18.51 in 1988)
provided compatibility with Common Lisp's \texttt{defstruct}, including the
ability to define new structs as extensions/subtypes of others, thus
providing a limited form of inheritance, actual support for object-oriented
programming in the form of method dispatch has been historically limited
in Emacs.

The first real step in that direction was the development of EIEIO by Eric
Ludlam around the end of 1995, beginning of 1996.  The official name ``Enhanced
Implementation of Emacs Interpreted Objects'' hints at the earlier existence
of some ``Emacs Interpreted Objects'' package but in reality the acronym
came before its expansion.  EIEIO started as an experiment to try use
an object system in Emacs, first following a model like that of C++, but
very soon switching to a CLOS-inspired model.

EIEIO is an implementation of a subset of CLOS.  Its development was mostly
driven by actual needs more than as an end in itself: the original
motivation was to try and play with an object request broker, then a widget
toolkit, and later switched to providing support for the CEDET package.
It included support for most of CLOS's \texttt{defclass}, as well as support
for a subset of \texttt{defmethod}, more specifically it was limited to
single-dispatch methods, dispatching on the first argument, and it could
only dispatch based on types of \texttt{defclass} objects.  It also had
incomplete support for method combinations, only allowing \texttt{:before}
and \texttt{:after} methods but not \texttt{:around} nor any user-defined
additional qualifiers.

EIEIO spent most of its life as part of the CEDET package (a package
providing IDE-like features) before being integrated into Emacs-23.2 in
2010, along with most of CEDET.  Use of EIEIO within Emacs stayed fairly
limited, partly for reasons of inertia, but also because EIEIO suffered some
of the same problems as \texttt{cl.el} in that it was not
``namespace clean''.

At the end of 2014, Stefan Monnier started to try and clean up EIEIO so as
to be able to use it in more parts of Emacs.  The intention was basically to
add a ``\texttt{cl-}'' prefix as was done for \texttt{cl-lib} (because it
was perceived that an ``\texttt{eieio-}'' prefix would be too verbose to be
popular), but there was also a desire to improve the \texttt{defmethod} with
support for \texttt{:around} methods and dispatch on other types than those
defined with \texttt{defclass}.

It became quickly evident that the implementation of method dispatch
needed a complete overhaul: rather than constructing combined methods
up-front and memoizing the result, as in typical CLOS implementations,
EIEIO's dispatch and \texttt{call-next-method} did all their work
dynamically, relying on dynamically-scoped variables to preserve state in
a way that was both brittle and somewhat inefficient.

So, instead of improving EIEIO's \texttt{defmethod}, a completely new
version of CLOS's \texttt{defmethod} was implemented in the new
\texttt{cl-generic.el} package, which appeared in Emacs-25.1.  The main
immediate downside was that the idea to cleanup the rest of EIEIO (which
implements \texttt{defclass} objects) ended up forgotten along the way.
The implementation is not super efficient, but it's already several times
faster than the previous one in EIEIO.  This package provides largely the
same featureset as CLOS's \texttt{defmethod}, except for some important
differences:
\begin{enumerate}
\item Method combinations cannot be specified per method like in CLOS, but
  instead new method combinations can be added globally by adding
  appropriate methods to \texttt{cl-\linebreak[0]generic-\linebreak[0]combine-\linebreak[0]methods}.  This seemed
  like a good idea, but there is no known user of this feature at this time,
  not even internal.
\item The set of supported specializers is not hard-coded.  Instead, they
  can be defined in a modular way via the notion of \emph{generalizer}
  inspired from~\cite{Rhodes14}.  This is used both internally (to define
  all the standard specializers) as well as in some external packages, most
  notably in EIEIO to support dispatching on \texttt{defclass} types.
\end{enumerate}
The main motivation for the first difference above was that CLOS's support
for method combinations seemed too complex: the cost of implementation was
not justified by the expected use of the feature, so it was replaced by
a much simpler mechanism.

As for the second difference, it was made necessary by the need to dispatch
on EIEIO objects even though \texttt{cl-generic.el} could not depend on
EIEIO since it was not clean enough.  There were additional motivations for
it, though: not only it was clearly desirable to be able to define new
specializers, but it also made the implementation of the main specializers
cleaner, and most importantly it seemed like an interesting problem
to solve.

%% FIXME Stefan: I elided this, as it's tangential to design
%% Just as in the CLOS MOP, there is a fair bit of delicate bootstrapping
%% involved: new specializers are defined by adding methods to the
%% \texttt{cl-generic-generalizers} generic function, so we first define the
%% \texttt{t} specializer in a somewhat ad-hoc way, of course, then we define
%% the \texttt{head} specializer (where $\texttt{(head \textsl{VAL})}$ means
%% that the method is only applicable if the argument's \texttt{car} is equal
%% to \textsl{VAL}), after which we use this new specializer to implement the
%% \texttt{eql} specializer.

Existing code that could make use of this new machinery,
required dispatching on contextual information
(i.e.\ on the current state) rather than only on arguments.
Consequently, \texttt{cl-generic.el} also adds support to its \texttt{cl-defmethod} for
pseudo-arguments of the form ``\texttt{\&context (\textsl{EXP}
  \textsl{SPECIALIZER})}''.  This is used for methods which are only
applicable in specific contexts, such as in specific major modes or in
frames using a particular kind of GUI.

The implementation of \texttt{cl-generic.el} was accompanied by an extension
of the on-line help system so as to be able to give information not just
about \Elisp{} variables, functions, and faces but also other kinds of named
elements, starting with types.  And to go along with that, the implementation
of \texttt{cl-defstruct} was improved to better preserve information about
the type hierarchy so that the on-line help system can be used to traverse
it.  This started as an attempt to adapt to \texttt{cl-generic.el} the EIEIO
facilities to explore interactively EIEIO objects and methods, but is more
modular and better integrated with the rest of Emacs's on-line help system.

\subsection{Actual objects}  %Emacs-26.1
\label{sec:actual-objects}

While Emacs-25's \texttt{cl-generic.el} introduced object-oriented
programming facilities into \Elisp{}, objects (whether defined via
\texttt{cl-lib}'s \texttt{cl-defstruct} or EIEIO's \texttt{defclass}) were
still represented as vectors and hence couldn't be reliably distinguished
from vectors, for example to pretty-print them.

This was addressed in Emacs-26 by the introduction of the
\texttt{make-record} primitive and corresponding new object type.
Those \emph{records} are implemented just like the vectors used previously,
except that their tag indicates that they should be treated as records
instead of vectors, and that by convention the first field of a record is
supposed to contain a type descriptor, which can be just a \emph{symbol}.

The main complexity introduced by this change was the need for a new syntax
to print and read those new objects, as well as the incompatibility between
the printed representation of objects using the old vector-based encoding
and those using the new encoding.

\subsection{Generators}
\label{sec:generators}

With the success of Python's and Javascript's iterators and generators, some
Emacs users felt like \Elisp{} was lacking in abstraction, so in 2015,
Daniel Colascione developed the \texttt{generator.el}, which was included
into Emacs-25.1.  It makes it easy and convenient to write generators using
macros \texttt{iter-lambda} and \texttt{iter-yield}.  Its implementation is
based on a kind of local conversion to continuation-passing style (CPS) and
hence relies extensively on the use of lexical scoping, to work around the
fact that \Elisp{} does not directly provide something like \texttt{call/cc}
to access underlying continuations.  It only handles a (relatively large)
subset of \Elisp{}, because CPS conversion of forms like
%% FIXME: Add citation to dynamic-wind or other Scheme article about
%% unwind-protect.
\texttt{unwind-protect} cannot be defined in general in \Elisp.

\subsection{Concurrency}
\label{sec:concurrency}

\Elisp{} is a fundamentally sequential language, and it relies very heavily
on side-effects to a global state.  Yet, its use in an interactive program
has inevitably lead to a desire for concurrency to try and improve
responsiveness.  So concurrency appeared very early on: already in
Emacs-16.56 Emacs included support for asynchronous processes, i.e.~the
execution of separate programs whose output was processed by so-called
\emph{process filters} whenever the \Elisp{} execution engine is idly
waiting for the next user command.

While this very limited form of cooperative concurrency was slightly
improved in 1994's Lucid Emacs 19.9 and 1996's Emacs-19.31 by adding native support for timers (they
were earlier implemented as an asynchronous process sending Emacs output at
the requested time), it has been the only form of concurrency available for
most of Emacs's life.

Adding true shared-memory concurrency to \Elisp{} is very problematic
because of the pervasive reliance on a shared state in all of existing
\Elisp{} code, but the limited existing support was limiting even in those
cases where it was technically sufficient: many \Elisp{} packages which
interact with external programs block many more times than really necessary,
simply because in order to avoid it the coder needs to write their code in
a \emph{continuation-passing style}, which interacts poorly with dynamic
scoping and requires significant surgery to retro-fit to existing code.

So, shared-memory concurrency was largely considered as inapplicable to
\Elisp{}.  Nevertheless, in November 2008, Giuseppe Scrivano posted a first
naive attempt at adding threads to \Elisp{}.  This effort did not go much
further, but it inspired Tom Tromey to try its own luck at the game.
In 2010, he started to work on adding shared-memory cooperative concurrency
primitives like \texttt{make-thread} to Emacs.  Interaction with the
implementation of dynamic scoping, which is based on a global state for
speed, required experimentation with various approaches.  Correctly handling
buffer-local and frame-local bindings without a complete rewrite was
particularly painful and many approaches were abandoned simply because it
was too difficult to keep them up-to-date with the evolving Emacs codebase.

This result of this was finally released in 2018, as part of Emacs-26.1.
Context switches still only take place at a few known points where \Elisp{}
is idle (or via explicit calls to \texttt{thread-yield}).  The current
implementation of this feature makes context switches take time proportional
to the current stack depth, because the dynamic bindings of the old thread
need to be saved and removed, after which the dynamic bindings of the new
thread need to be restored.  Earlier implementation approaches tried to
avoid this expensive form of context switching, by making global variable
lookups a bit more expensive instead, but these required much more extensive
and delicate changes to existing code, so while they may still be good
options for the future, this first implementation favored a simpler and
safer approach.

Over the years, other approaches to concurrency and parallelism have been
developed as \Elisp{} packages, most notably the \texttt{async.el} package
developed in 2012 which runs \Elisp{} code in parallel in a separate
Emacs subprocess.

\subsection{Inline functions}
\label{sec:inline-functions}

Function calls are fairly expensive in \Elisp{}, and their semantics
involves looking up the current definition in the global name space, so
function inlining is at the same time important for performance and
semantically visible.

So during the development of the new byte-code compiler for Emacs-19, a new
\texttt{defsubst} macro was added which works like \texttt{defun} except
that it annotates the function so the byte-code compiler inlines it whenever it
can.  This inlining was fairly naive, but worked both for compiled and
non-compiled functions (by either inlining the function's body into the
source code or into the generated byte-code of the caller).

The new \texttt{cl.el} package by Dave Gillespie included in 1993 introduced
a new form of inlining in the form of the \texttt{defsubst*} macro, which is
almost identical to \texttt{defsubst} from the outside, except for including
support for Common Lisp extensions like keyword arguments, but its
implementation is different and does not correctly preserve the semantics of
the dynamically scoped formal arguments, which are instead replaced by
substitution with the actual arguments.

Of course, additionally to those two, \Elisp{} came with a third way to
implement an ``inlinable function'', namely by defining it as a macro.

In late 2014, while working to adapt the \texttt{cl-lib} library to the
changes in EIEIO and \texttt{cl-defstruct} objects, Stefan Monnier became
frustrated by the redundancy between \texttt{cl-typep}'s definition and its
compiler macro.  The \texttt{cl-typep} function takes two arguments and tests if the first is
a value of the type specified by the second.  The function definition takes
care of the general case where the type argument is only known at run-time,
but the compiler-macro is important to optimize for example
\texttt{(cl-typep x 'integer)} into \texttt{(integerp x)}.  While this
optimization could arguably be performed automatically by a sufficiently
sophisticated compiler, the \Elisp{} compiler is much too naive for that.
So he developed the new macro \texttt{define-inline}, which was included in
Emacs-25.1.  It lets one define a function in such a way that from its
definition the macro can extract both a normal function body and
a corresponding compiler-macro.

%% Daniel Colascione: Do you want to mention how Emacs Lisp is _defined_ to
%% expand compiler macros, which, AIUI, distinguishes it from other lisps?
%% Stefan: I don't think it's defined to do that (after all, it only happens
%% in `macroexpand-all` but not in `macroexpand` so it doesn't happen
%% for interpreted code that's not eagerly macroexpanded).

\subsection{Module system}
\label{sec:module-system}

While \Elisp{} was designed from the beginning as a real programming
language rather than a tiny ad-hoc extension language, it was not designed
for ``programming in the large'' as witnessed by the lack of module or
namespace system.

Yet, the \Elisp{} side of Emacs is now a rather large system, so in order to
avoid name conflicts, \Elisp{} relies on a poor man's namespace system, as
mentioned in Sec.~\ref{sec:lexical-scoping}, where code loosely follows
a convention where global functions and variables belonging to package
\textsl{<pkg>} will define identifiers starting with a \texttt{<pkg>-}
prefix (and use a prefix of \texttt{<pkg>--} in order to indicate that this
identifier should be considered an internal definition).

There have been many attempts to remedy this situation by providing support
for some form of namespacing:
\begin{itemize}
\item In May 2011, Christopher Wellons developed the Fakespace package which
  lets one define private variables and functions and then prevents them
  from escaping into the global namespace.  Non-private definitions still
  rely on the usual package-prefix naming convention to avoid conflicts.
\item In October 2012, Chris Barrett developed the Namespaces package which
  provides an extensive set of new macros to define namespaces, define
  functions and variables in those namespaces, and use them from
  other namespaces.
\item In March 2013, Wilfred Hughes developed the proof-of-concept
  \texttt{with-namespace} macro which basically adds the specified namespace
  prefix to all the elements defined in its body.
\item At the same time, Yann Hodique developed the proof-of-concept Codex
  package which instead tries to provide a functionality similar to
  Common Lisp's packages, where each package has its own \emph{obarray}.
\item In early 2014, Artur Malabarba developed the Names package, which
  takes the approach of \texttt{with-namespace}, but doing a much more
  thorough job.
\item In 2015, the same Artur Malabarba developed the Nameless package which
  takes a completely different approach: it does not provide any new
  \Elisp{} construct and instead focuses on making Emacs \emph{hide} the
  package prefixes from the user while working on the code.
\end{itemize}
%%
To date, no namespacing facility has been incorporated into Emacs, nor seen
much use in other packages.  The last time this subject was (hotly) debated
among Emacs maintainers and contributors, around 2013 following a blog post
by Nic Ferrier, no consensus appeared, but other than inertia one of the
main arguments in favor of the status quo was that \Elisp's poor man's
namespacing makes cross-referencing easier: any simple textual search can be
used to find definitions and uses of any global function or variable,
including filesystem-wide searches or even web searches,
whereas all the alternatives introduce names that need to be interpreted
relative to their context forcing the reliance on an IDE that understands the
particular namespacing used when browsing the code.

%% FIXME: Should we talk about "package systems" in XEmacs / ELPA etc.?

%% FIXME: Compilation of Elisp to C by Tom Tromey!

\section{Alternative implementations}
\label{sec:alternative-implementations}

Implementation of \Elisp{} have not been confined to Emacs and its
derivatives.  Two implementations---Edwin and JEmacs---are notable for
running \Elisp code on editors implemented independently from Emacs.
Moreover, a Common Lisp package emulates Emacs Lisp, and Guile Scheme
also comes with support for the \Elisp{} language.
These implementations all aim at running existing \Elisp{}
code in alternative environments, and consequently feature no
significant language changes.

\subsection{Edwin}

Edwin is the editor that ships with MIT Scheme~\cite{MITScheme2014}.
Its user interface is based on that of Emacs.  Edwin is implemented
completely in Scheme, and Scheme is its native extension language.
Additionally, Matthew Birkholz implemented an \Elisp{} interpreter
in Scheme that was able to run substantial \Elisp{}
packages~\cite{Birkholz1993} at the time, among them the Gnus news reader.

\subsection{Librep}

In 1993, John Harper started working on an embeddable implementation of
\Elisp{} called Librep, which is most famously used as the extension
language of the Sawfish window manager.  While Librep started as a Lisp
dialect that was mostly compatible with \Elisp{}, it has since significantly
diverged, including a module system, lexical scoping, tail-call elimination,
and first-class continuations.

\subsection{Elisp in Common Lisp}

%% FIXME: Ask Sam what were his motivations for this work and how complete
%% it is.
In 1999, Sam Steingold also implemented the \Elisp{} language as a Common
Lisp package~\cite{Steingold99}.  He was motivated by the hope of moving
Emacs to work with a Common Lisp language instead of \Elisp{}.
His \texttt{elisp.lisp} package does not attempt to reimplement the library
of functions provided in Emacs to manipulate buffers and other related
objects, so it focuses on the ``pure'' \Elisp{} language; but it was able to
run the non-UI parts of the Emacs Calendar, which provides sophisticated
functions to manipulate and convert dates between many historical calendars.

\subsection{JEmacs}

JEmacs~\cite{Bothner2001} is an editor that ships with Kawa
Scheme~\cite{KawaScheme}.  JEmacs comes with support for running some
\Elisp{} code.  Its implementation (written partly in Java and
partly in Scheme) works by translating \Elisp{} code to Scheme, and
running the result.

\subsection{Guile}

Guile Scheme~\cite{Guile2018} was conceived as the universal extension
language of the GNU project, with the specific intention of replacing
\Elisp in Emacs at one point.  This has not happened (yet), but Guile does
ship with a fairly complete implementation of \Elisp{} that translates
\Elisp{} programs to Guile's intermediate language.  It is used in the
Guile-Emacs system, which is a work-in-progress modification of Emacs where
the \Elisp{} engine is provided by Guile.
%% FIXME: Motivations?

\subsection{Emacs-Ejit}

In 2013, Nic Ferrier implemented in \Elisp{} a compiler from \Elisp{} to
Javascript.  This was designed so as to be able to write complete web sites
all in \Elisp{}, using the Elnode \Elisp{} package to do the server-side
processing and using Emacs-Ejit to write the client-side code in \Elisp{}
as well.  It does not really aim to run any \Elisp{} package in your
browser, so its runtime library only provides a small subset of \Elisp's
standard primitives.

\section{Conclusion}
\label{sec:conclusion}

Many steps of \Elisp's evolution have been the result of efforts of single
individuals, driven by a specific purpose, yet it has managed to keep an
arguably sane and cohesive overall design, thanks on the one hand to
a maintainership which was more interested in improving the text editor than
the language and kept an eye on the longer term, and on the other to the
willingness to break backward compatibility in specific cases, in order to
gradually address problems encountered over time.

Like Lisp in general the development of Emacs has seen plenty of
discord, with unbridgeable personal differences, heated debates, and
forks along the way.  However, througout these differences, \Elisp{}
has steered a remarkably stable course of conservative development and
gradual extension.  It has mostly grown by slowly incorporating popular
features from other languages, both in the language itself and in its
implementation.  But it has also come up with its own features, such as
docstrings, buffer-local variables, and the addition of text-properties
to strings.

Possibly, the organic growth of all the \Elisp{} packages developed within
the community of Emacs users and developers, whose composability relies on
social mechanisms, has provided enough cohesive force to keep balkanization
at bay for more than 30 years.  Given the amount of changes it went through
over the last decade, we are looking forward to the \Elisp{} of the next
30 years.

%% FIXME: Discuss whether \Elisp{} is more of a sponge
%% that incorporated other languages's features, or one that comes up
%% with its own solutions which are sometimes then picked up by
%% other languages?

\subsection{Acknowledgments}

Emacs and \Elisp{} are the result of the contribution of an impressive
number of individuals.  We thank them all for their contributions of course.
With respect to this article, while the efforts put into maintaining the
revision history of Emacs through the various revision systems it has used
have been very helpful, we'd like to thank also Lars Brinkhoff for his
archiving work at \url{https://github.com/larsbrinkhoff/emacs-history} which
fills some of the holes of the early life of Emacs.

\bibliographystyle{abbrvnat}
\bibliography{refs}

\end{document}
